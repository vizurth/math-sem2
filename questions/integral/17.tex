{
\subsection{Несобственные интегралы II рода: определение, главное значение. Критерий сходимости интеграла II рода от неотрицательной функции. }

\subsection*{Определение 2.1}

1) Пусть \( f(x) \) определена на \( (a, b] \) и не ограничена в любой правой полукрестности точки \( a \). Пусть \( f(x) \in R([\alpha, b]) \) \( \forall \alpha \in (a, b] \).

\[
\int_{a}^{b} f(x)dx
\]

Символ \( \int_{a}^{b} f(x)dx \) называется несобственным интегралом II рода.

Если существует конечный предел \( I = \lim_{\alpha \to a+0} \int_{\alpha}^{b} f(x)dx \), то символу \( \int_{a}^{b} f(x)dx \) приписывают значение \( I \), то есть,



\[
\int_{a}^{b} f(x)dx = \lim_{\alpha \to a+0} \int_{\alpha}^{b} f(x)dx
\]



и говорят, что несобственный интеграл \textbf{сходится}.

Если предел бесконечен или не существует, то говорят, что несобственный интеграл \textbf{расходится}.

2) Пусть \( f(x) \) определена на \( [a, b) \) и не ограничена в любой левой полукрестности точки \( b \). Пусть \( f(x) \in R([a, \beta)) \) \( \forall \beta \in [a, b) \).

Символ \( \int_{a}^{b} f(x)dx \) называется несобственным интегралом II рода.

Если существует конечный предел 



\[
I = \lim_{\beta \to b-0} \int_{a}^{\beta} f(x)dx,
\]



то символу \( \int_{a}^{b} f(x)dx \) приписывают значение \( I \), то есть,



\[
\int_{a}^{b} f(x)dx = \lim_{\beta \to b-0} \int_{a}^{\beta} f(x)dx
\]



и говорят, что несобственный интеграл \textbf{сходится}.

Если предел бесконечен или не существует, то говорят, что несобственный интеграл \textbf{расходится}.

3) Пусть \( f(x) \) определена на \( [a, b] \) всюду, за исключением точки \( c \in (a, b) \), и не ограничена в любой окрестности точки \( c \).

Пусть \( f(x) \) интегрируема на любом отрезке, содержащемся в \( [a, b] \) и не содержащем точку \( c \).



\[
\int_{a}^{b} f(x)dx
\]



В этом случае символ \( \int_{a}^{b} f(x)dx \) также называется несобственным интегралом II рода.

Есть два равносильных способа приписать символу \( \int_{a}^{b} f(x)dx \) числовое значение:
a)


\[
\int_{a}^{b} f(x)dx = \int_{a}^{c} f(x)dx + \int_{c}^{b} f(x)dx
\]

\[
\int_{a}^{b} f(x)dx \text{ сходится, если } \int_{a}^{c} f(x)dx \text{ и } \int_{c}^{b} f(x)dx \text{ сходятся.}
\]

б)

\[
\int_{a}^{b} f(x)dx = \lim_{\delta_{1} \to 0, \delta_{2} \to 0} \left( \int_{a}^{c-\delta_{1}} f(x)dx + \int_{c+\delta_{2}}^{b} f(x)dx \right).
\]


\subsection*{Замечание}

Если не существует конечный 



\[
\lim_{\delta_1 \to +0} \left( \int_{a}^{c-\delta_1} f(x)dx \right) + \int_{c+\delta_2}^{b} f(x)dx,
\]



но существует конечный 



\[
\lim_{\delta \to +0} \left( \int_{a}^{c-\delta} f(x)dx + \int_{c+\delta}^{b} f(x)dx \right),
\]



то этот предел называют \textbf{главным значением интеграла}



\[
\int_{a}^{b} f(x)dx
\]



и обозначают \textbf{v.p.}



\[
\int_{a}^{b} f(x)dx.
\]



(то есть, 



\[
v.p. \int_{a}^{b} f(x)dx = \lim_{\delta \to +0} \left( \int_{a}^{c-\delta} f(x)dx + \int_{c+\delta}^{b} f(x)dx \right).
\]

\subsection*{Лемма 2.1}

Рассмотрим интеграл из пункта 2) определения 2.1 (для интегралов из пунктов 1) и 3) аналогично).

1) Пусть \( a' \in (a, b) \). Тогда



\[
\int_{a}^{b} f(x) dx \text{ сходится} \Leftrightarrow \int_{a}^{a'} f(x) dx \text{ сходится} \text{ и } \int_{a'}^{b} f(x) dx \text{ сходится}
\]

\[
\text{(и }\int_{a}^{b} f(x) dx = \int_{a}^{a'} f(x) dx + \int_{a'}^{b} f(x) dx).
\]

2) Пусть \( c \neq 0 \). Тогда

\[
\int_{a}^{b} c f(x) dx \text{ сходится} \Leftrightarrow \int_{a}^{b} f(x) dx \text{ сходится}
\]
\[
\text{(и }\int_{a}^{b} c f(x) dx = c \int_{a}^{b} f(x) dx).
\]

\subsection*{Теорема 2.1}

Критерий сходимости несобственного интеграла II рода от неотрицательной функции.

Рассмотрим интеграл из пункта 2) определения 2.1 (для интегралов из пунктов 1) и 3) аналогично).

Пусть \( f(x) \geq 0 \) на \( [a, b) \). Тогда

\[
\int_{a}^{b} f(x) dx \text{ сходится } \Leftrightarrow \exists K \geq 0: \int_{a}^{\beta} f(x) dx \leq K \quad \forall \beta \in [a, b).
\]

\subsection*{Лемма 2.2}

Пусть \( F(x) \) возрастает на \( [a, b) \). Тогда

\[
\lim_{x \to b-0} F(x) \text{ конечный } \Leftrightarrow F(x) \text{ ограничена сверху на } [a, b).
\]

(без доказательства)

}