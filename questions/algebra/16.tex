{
\subsection{
Собственные числа и векторы матрицы. Совпадение характеристических многочленов у подобных матриц. Линейная независимость собственных векторов, соответствующих различным собственным числам. 
}

\subsection*{Определение собственных чисел и векторов}

Пусть $A$ — квадратная матрица размера $n \times n$. Число $\lambda$ называется \textbf{собственным числом} матрицы $A$, если существует ненулевой вектор $x$, удовлетворяющий уравнению:



\[
A x = \lambda x.
\]



Такой вектор $x$ называется \textbf{собственным вектором}, соответствующим собственному числу $\lambda$.

\subsection*{Характеристический многочлен и его свойства}

Для нахождения собственных чисел рассматривают \textbf{характеристическое уравнение}:



\[
\det(A - \lambda E) = 0.
\]



Выражение $\chi_A (\lambda) = \det(A - \lambda E)$ называется \textbf{характеристическим многочленом} матрицы $A$. Корни этого многочлена — собственные числа матрицы.

\subsection*{Собственные числа подобных матриц}

Две матрицы $A$ и $B$ называются \textbf{подобными}, если существует невырожденная матрица $S$, такая что:



\[
B = S^{-1} A S.
\]



Подобные матрицы имеют одинаковые характеристические многочлены:



\[
\det(B - \lambda E) = \det(S^{-1} A S - \lambda E).
\]



С учетом свойства определителя:



\[
\det(S^{-1} (A - \lambda E) S) = \det(A - \lambda E),
\]



откуда следует, что характеристический многочлен матрицы $B$ совпадает с характеристическим многочленом матрицы $A$, а значит, подобные матрицы имеют одинаковые собственные числа.

\subsection*{Линейная независимость собственных векторов, соответствующих различным собственным числам}

Если $\lambda_1, \lambda_2, \dots, \lambda_m$ — различные собственные числа матрицы $A$, а $x_1, x_2, \dots, x_m$ — соответствующие собственные векторы, то система векторов $\{x_1, x_2, \dots, x_m\}$ \textbf{линейно независима}.

Доказательство:

Рассмотрим произвольную линейную комбинацию:



\[
c_1 x_1 + c_2 x_2 + \dots + c_m x_m = 0.
\]



Применим матрицу $A$ к этому равенству:



\[
A (c_1 x_1 + c_2 x_2 + \dots + c_m x_m) = c_1 A x_1 + c_2 A x_2 + \dots + c_m A x_m.
\]



По определению собственных векторов:



\[
c_1 \lambda_1 x_1 + c_2 \lambda_2 x_2 + \dots + c_m \lambda_m x_m = 0.
\]



Вычтем из него исходное уравнение:



\[
c_1 (\lambda_1 - \lambda) x_1 + c_2 (\lambda_2 - \lambda) x_2 + \dots + c_m (\lambda_m - \lambda) x_m = 0.
\]



Так как собственные числа различны, коэффициенты $\lambda_i - \lambda$ ненулевые. Следовательно, из линейной независимости векторов следует, что $c_1 = c_2 = \dots = c_m = 0$.

Таким образом, собственные векторы, соответствующие различным собственным числам, линейно независимы.

}