{
\subsection{ Определение точек условного экстремума функции нескольких переменных. Необходимые и достаточные   условия существования точек условного экстремума. Пример: найти точки условного экстремума функции \( f(x,y) = x^3 + y^3 - 9xy\)   при условии \(x + y = 0\)  ,  используя метод нахождения точек условного экстремума.}

\subsection*{Пример}

1) Найдем экстремумы функции \( f(x, y) = x^3 + y^3 - 9xy + 5 \) при условии, что переменные \( x \) и \( y \) связаны соотношением \( x + y = 0 \).

Отсюда \( y = -x \), и \( f = 9x^2 + 5 \).

\( f \) имеет минимум в точке \( A(0,0) \) при условии \( x + y = 0 \).

2) \( f \) имеет максимум в точке \( A(0,0) \) при условии \( x - y = 0 \) (проверьте самостоятельно).

Такие экстремумы называются \textbf{условными}.

\subsection*{Определение 7.1}

Пусть \( \Omega \subseteq \mathbb{R}^p \), \( f: \Omega \to \mathbb{R} \). Рассматриваем только точки, удовлетворяющие условиям связи:



\[
\begin{cases}
\varphi_1(X) = 0 \\
\varphi_2(X) = 0 \\
\vdots \\
\varphi_m(X) = 0
\end{cases}
\quad (m < p).
\]



Пусть \( A \) - внутренняя точка множества \( \Omega \), удовлетворяющая условиям связи.

Точка \( A \) называется \textbf{точкой условного максимума (минимума)} функции \( f \), если существует окрестность \( \tilde{U}(A) \subseteq \Omega \), такая что:



\[
f(X) \leq f(A) \quad (\text{или} \quad f(X) \geq f(A))
\]



для всех точек \( X \), которые принадлежат \( \tilde{U}(A) \) и удовлетворяют условиям связи.

Если неравенства строгие, такие точки называются \textbf{точками строгого условного экстремума} функции \( f \).

\subsection*{Теорема 7.1. Необходимый признак условного экстремума}

Пусть \( f(X) \) и \( \Phi_i(X) \) \( (i=1,2,...,m) \) непрерывно дифференцируемы в точке \( A \), и матрица Якоби



\[
\begin{pmatrix}
\frac{\partial \Phi_1}{\partial x_1} & \frac{\partial \Phi_1}{\partial x_2} & \ldots & \frac{\partial \Phi_1}{\partial x_p} \\
\vdots & \vdots & \ddots & \vdots \\
\frac{\partial \Phi_m}{\partial x_1} & \frac{\partial \Phi_m}{\partial x_2} & \ldots & \frac{\partial \Phi_m}{\partial x_p}
\end{pmatrix}
\]



имеет ранг \( m \) в точке \( A \).

Тогда, если \( A \) - точка условного экстремума функции \( f \) при условиях связи \( \Phi_i(X) = 0 \) \( (i=1,2,...,m) \), то существуют числа \( \lambda_1, \lambda_2, ..., \lambda_m \) такие, что



\[
\lambda_1 \frac{\partial \Phi_1}{\partial x_j}(A) + ... + \lambda_m \frac{\partial \Phi_m}{\partial x_j}(A) + \frac{\partial f}{\partial x_j}(A) = 0 \quad (j=1,2,...,p).
\]




\subsection*{Пример 1}

Найдем экстремумы функции \( f(x, y) = x^3 + y^3 - 9xy + 5 \) при условии, что переменные \( x \) и \( y \) связаны соотношением \( x + y = 0 \).

Рассмотрим функцию Лагранжа \( F(x, y, \lambda) = x^3 + y^3 - 9xy + 5 + \lambda (x + y) \).



\[
\begin{cases}
F'_x = 0 \\
F'_y = 0 \\
условие связи
\end{cases}
\Leftrightarrow
\begin{cases}
3x^2 - 9y + \lambda = 0 \\
3y^2 - 9x + \lambda = 0 \\
x + y = 0
\end{cases}
\Rightarrow
\begin{cases}
x = 0 \\
y = 0 \\
\lambda = 0
\end{cases}
\]



б). \( F''_{xx} = 6x, \, F''_{xy} = -9, \, F''_{yy} = 6y \).

В точке \( A(0,0) \) \( d^2F(A) = -18dxdy \), где приращения \( dx \) и \( dy \) связаны соотношением \( d\varphi(A, \Delta X) = 0 \Leftrightarrow dx + dy = 0 \Leftrightarrow dy = -dx \).

Следовательно, \( d^2F(A) = 18dx^2 > 0 \, \forall \, dx, \, не \, равного \, 0 \).

Следовательно, точка \( A(0,0) \) - точка условного минимума функции \( f \) при условии \( x + y = 0 \).

\subsection*{Теорема 7.2. Достаточный признак условного экстремума}

Пусть \( f(X) \) и \( \Phi_i(X) \) \( (i=1,2,...,m) \) дважды непрерывно дифференцируемы в точке \( A \), и матрица Якоби



\[
\begin{pmatrix}
\frac{\partial \Phi_1}{\partial x_1} & \frac{\partial \Phi_1}{\partial x_2} & \ldots & \frac{\partial \Phi_1}{\partial x_p} \\
\vdots & \vdots & \ddots & \vdots \\
\frac{\partial \Phi_m}{\partial x_1} & \frac{\partial \Phi_m}{\partial x_2} & \ldots & \frac{\partial \Phi_m}{\partial x_p}
\end{pmatrix}
\]



имеет ранг \( m \) в точке \( A \).

Пусть точка \( A \) удовлетворяет необходимому признаку условного экстремума, то есть существуют числа \( \lambda_1, \lambda_2, ..., \lambda_m \), такие что



\[
\lambda_1 \frac{\partial \Phi_1}{\partial x_j}(A) + ... + \lambda_m \frac{\partial \Phi_m}{\partial x_j}(A) + \frac{\partial f}{\partial x_j}(A) = 0 \quad (j=1,2,...,p).
\]



Рассмотрим второй дифференциал функции Лагранжа:



\[
d^2F(A, \Delta X) = d^2f(A, \Delta X) + \lambda_1 d^2\Phi_1(A, \Delta X) + ... + \lambda_m d^2\Phi_m(A, \Delta X).
\]



Если эта квадратичная форма положительно (отрицательно) определена, то точка \( A \) является точкой строгого условного минимума (максимума) функции \( f \).


}