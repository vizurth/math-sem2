{
\subsection{Дифференцируемость функции. Формула Ньютона – Лейбница.}
\subsection*{Следствия}

1) Частный случай (теорема Барроу):

Пусть \( f(x) \in C([a, b]) \). Тогда \( F'(x) = f(x) \) на \( [a, b] \).

(То есть, у любой непрерывной на отрезке функции существует первообразная.)

2) Формула Ньютона-Лейбница:

Пусть \( f(x) \in C([a, b]) \), \( F(x) \) — некоторая первообразная функции \( f(x) \) на \( [a, b] \).

Тогда

\[
\int_{a}^{b} f(x)dx = F(b) - F(a).
\]

\subsection*{Доказательство}

2) Так как \( \Phi(x) = \int_{a}^{x} f(t)dt \) также является первообразной функции \( f(x) \) на \( [a,b] \), то существует число \( c \): 


\[
\Phi(x) = \int_{a}^{x} f(t)dt = F(x) + c,
\]


то есть,



\[
\int_{a}^{x} f(t)dt = F(x) + c \quad \forall \, x \in [a,b].
\]



Пусть \( x = a \).


\[
0 = F(a) + c.
\]


Следовательно, \( c = -F(a) \). Пусть \( x = b \).



\[
\int_{a}^{b} f(x)dx = F(b) - F(a).
\]
}