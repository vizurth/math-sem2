{
\subsection{Дифференциалы высших порядков. Отсутствие инвариантности формы у дифференциалов порядка выше первого.}

\subsection*{Определение 4.2}

Пусть \( f \) определена и дифференцируема в окрестности \( U(A) \subset \mathbb{R}^p \). Рассмотрим \( df(X) \), определенный в \( U(A) \), как функцию от \( X \) (приращение \( dX = (dx_1, ..., dx_p) \) считаем фиксированным). Если существует дифференциал этой функции \( df(X) \) в точке \( A \), он называется вторым дифференциалом функции \( f \) в точке \( A \) и обозначается \( d^2f(A) \).

Аналогично \( d^3f(A) = d(d^2f)(A), \ldots, d^kf(A) = d(d^{k-1}f)(A) \).

Все дифференциалы считаются при одном и том же приращении \( dX = (dx_1, ..., dx_p) \).

\subsection*{Теорема 4.2}

Пусть \( f(X) \) имеет в окрестности \( U(A) \subset \mathbb{R}^p \) все частные производные до \( k \)-го порядка включительно, и они непрерывны в точке \( A \). Тогда



\[
\exists \, d^kf(A) = d^kf(A, dX) = \left( \frac{\partial}{\partial x_1} dx_1 + \ldots + \frac{\partial}{\partial x_p} dx_p \right)^k f(A)
\]



(здесь \( dX = (dx_1, ..., dx_p) \)).


\subsection*{Доказательство для \( p = 3 \), \( \kappa = 2 \)}

В точке \( A \) имеем:



\[
d^2 f = d (df) = d \left( f'_{x_1} dx_1 + f'_{x_2} dx_2 + f'_{x_3} dx_3 \right).
\]



Дифференцируем каждое слагаемое:



\[
d^2 f = d \left( f'_{x_1} \right) dx_1 + d \left( f'_{x_2} \right) dx_2 + d \left( f'_{x_3} \right) dx_3.
\]



Теперь раскрываем дифференциалы:



\[
d^2 f = \left( f''_{x_1 x_1} dx_1 + f''_{x_1 x_2} dx_2 + f''_{x_1 x_3} dx_3 \right) dx_1 +
\]





\[
+ \left( f''_{x_2 x_1} dx_1 + f''_{x_2 x_2} dx_2 + f''_{x_2 x_3} dx_3 \right) dx_2 +
\]





\[
+ \left( f''_{x_3 x_1} dx_1 + f''_{x_3 x_2} dx_2 + f''_{x_3 x_3} dx_3 \right) dx_3.
\]



Группируем слагаемые:



\[
d^2 f = f''_{x_1 x_1} dx_1^2 + f''_{x_2 x_2} dx_2^2 + f''_{x_3 x_3} dx_3^2 +
\]





\[
+ 2 f''_{x_1 x_2} dx_1 dx_2 + 2 f''_{x_1 x_3} dx_1 dx_3 + 2 f''_{x_2 x_3} dx_2 dx_3.
\]



Таким образом:



\[
d^2 f = \left( \frac{\partial}{\partial x_1} dx_1 + \frac{\partial}{\partial x_2} dx_2 + \frac{\partial}{\partial x_3} dx_3 \right)^2 f.
\]


\subsection*{Следствие: инвариантность}

Дифференциалы порядка выше первого не обладают свойством инвариантности формы.

Например, если \( x_1, x_2 \) не являются независимыми переменными, то:



\[
d^2 f(x_1, x_2) = d (df) = d \left( f'_{x_1} dx_1 + f'_{x_2} dx_2 \right) =
\]





\[
= d \left( f'_{x_1} dx_1 \right) + f'_{x_1} d (dx_1) + d \left( f'_{x_2} dx_2 \right) + f'_{x_2} d (dx_2) =
\]





\[
= \left( f''_{x_1 x_1} dx_1 + f''_{x_1 x_2} dx_2 \right) dx_1 +
\]





\[
+ \left( f''_{x_2 x_1} dx_1 + f''_{x_2 x_2} dx_2 \right) dx_2 + f'_{x_1} d^2 x_1 + f'_{x_2} d^2 x_2 =
\]





\[
= f''_{x_1 x_1} dx_1^2 + f''_{x_2 x_2} dx_2^2 + 2 f''_{x_1 x_2} dx_1 dx_2 + f'_{x_1} d^2 x_1 + f'_{x_2} d^2 x_2 =
\]





\[
= \left( \frac{\partial}{\partial x_1} dx_1 + \frac{\partial}{\partial x_2} dx_2 \right)^2 f + f'_{x_1} d^2 x_1 + f'_{x_2} d^2 x_2.
\]




}