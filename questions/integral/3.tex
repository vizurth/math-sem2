{
\subsection{Интегрирование с помощью замены переменной. Вычисление }
\subsection*{Теорема: Простейшие свойства неопределенного интеграла}

Пусть \( F(x) \) дифференцируема на \( \langle a, b \rangle \). Тогда 


\[
\int dF(x) = F(x) + c.
\]



Пусть существует \( \int f(x)dx \). Тогда 


\[
d\left(\int f(x)dx\right) = f(x)dx, \quad \text{то есть} \quad \left(\int f(x)dx\right)' = f(x) \quad \text{на} \quad \langle a, b \rangle.
\]



Пусть существуют \( \int f_1(x)dx \), \( \int f_2(x)dx \) на \( \langle a, b \rangle \). Тогда существует 


\[
\int (a f_1(x) + b f_2(x))dx = a \int f_1(x)dx + b \int f_2(x)dx \quad \text{на} \quad \langle a, b \rangle.
\]

\subsection*{Теорема 1.4: Интегрирование при помощи замены переменной}

Пусть существует


\[
\int f(t) dt = F(t) + c \quad \text{на } \langle a, b \rangle.
\]



Пусть \( \varphi(x) \) дифференцируема на \( \langle \alpha, \beta \rangle \), \( \varphi(\langle \alpha, \beta \rangle) = \langle a, b \rangle \). Тогда существует


\[
\int f(\varphi(x)) \varphi'(x) dx = F(\varphi(x)) + c \quad \text{на } \langle \alpha, \beta \rangle.
\]

\subsection*{Теорема 1.5: Интегрирование при помощи замены переменной (подстановка)}

Пусть существует


\[
\int f(\varphi(t)) \varphi'(t) dt = G(t) + c \quad \text{на } \langle \alpha, \beta \rangle.
\]



Пусть \( \varphi(t) \) дифференцируема и строго монотонна на \( \langle \alpha, \beta \rangle \), \( \varphi(\langle \alpha, \beta \rangle) = \langle a, b \rangle \).

Тогда существует

\[
\int f(x)dx = G(\varphi^{-1}(x)) + c \quad \text{на } \langle a, b \rangle.
\]

\subsection*{Пример:}

\[
\int \sqrt{a^2 - x^2} \,dx = \frac{a^2}{2} \arcsin \frac{x}{a} + \frac{x}{2} \sqrt{a^2 - x^2} + c \quad (a > 0).
\]

}