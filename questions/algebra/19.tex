{
\subsection{Ядро и образ отображения.}

\subsection*{Теорема 6.1}

Образ \( \operatorname{Im} A \) является подпространством линейного пространства \( W \).

Ядро \( \operatorname{Ker} A \) является подпространством линейного пространства \( V \).

\subsection*{Доказательство}

1) Пусть \( y_1, y_2 \in \operatorname{Im} A \), то есть существуют такие \( x_1, x_2 \in V \), что \( A x_1 = y_1 \), \( A x_2 = y_2 \).

Рассмотрим вектор \( y = a y_1 + b y_2 \). Он является образом вектора \( x = a x_1 + b x_2 \), так как:



\[
A (a x_1 + b x_2) = a A x_1 + b A x_2 = a y_1 + b y_2 = y.
\]



Следовательно, множество \( \operatorname{Im} A \) замкнуто относительно операций сложения и умножения на скаляр, что означает, что \( \operatorname{Im} A \) является подпространством \( W \).

2) Пусть \( x_1, x_2 \in \operatorname{Ker} A \), то есть \( A x_1 = 0 \), \( A x_2 = 0 \).

Рассмотрим вектор \( x = a x_1 + b x_2 \), тогда:



\[
A (a x_1 + b x_2) = a A x_1 + b A x_2 = 0.
\]



Следовательно, множество \( \operatorname{Ker} A \) замкнуто относительно операций сложения и умножения на скаляр, что означает, что \( \operatorname{Ker} A \) является подпространством \( V \).

\subsection*{Определение 6.3}

Пусть \( A : V \to W \) — линейное отображение, а \( e = \{e_1, e_2, \dots, e_n\} \) и \( f = \{f_1, f_2, \dots, f_m\} \) — базисы пространств \( V \) и \( W \).

**Матрицей линейного отображения \( A \) в базисах \( e \), \( f \)** называется матрица \( A \) размера \( m \times n \), столбцами которой являются координатные столбцы векторов \( A e_1, A e_2, \dots, A e_n \), то есть образов векторов \( e_1, e_2, \dots, e_n \) в базисе \( f \).

\subsection*{Замечание 11.1}

Для каждого \( j \) выполняется:



\[
A e_j = f A_{*j}
\]



(по замечанию 5.1).

Следовательно,



\[
(A e_1, A e_2, \dots, A e_n) = (f A_{*1}, f A_{*2}, \dots, f A_{*n}) \Rightarrow A e = f A.
\]



Здесь \( A e = A (e_1, e_2, \dots, e_n) = (A e_1, A e_2, \dots, A e_n) \), а \( f A \) — матричное произведение базисной строки \( f = (f_1, f_2, \dots, f_m) \) на матрицу \( A \).

\subsection*{Замечание 11.2}

Каждую матрицу \( A \) размера \( n \times n \) можно рассматривать как матрицу некоторого линейного оператора в некотором базисе.

\subsection*{Теорема 11.2}

Пусть \( A : V \to W \) — линейное отображение, \( e = \{e_1, e_2, \dots, e_n\} \) и \( f = \{f_1, f_2, \dots, f_m\} \) — базисы \( V \) и \( W \).

Матрица \( A \) размера \( m \times n \) является матрицей линейного отображения \( A \) в базисах \( e, f \).

Тогда \( \forall x \in V, \forall y \in W \) справедливо:



\[
A x = y \iff A X = Y
\]



(здесь \( X \) — координатный столбец вектора \( x \) в базисе \( e \), \( Y \) — координатный столбец вектора \( y \) в базисе \( f \)).

\subsection*{Теорема 6.3}

Пусть \( A : V \to W \) — линейное отображение, \( e, e' \) — базисы пространства \( V \), \( f, f' \) — базисы пространства \( W \).

Матрица \( A \) является матрицей линейного отображения \( A \) в базисах \( e, f \).

Матрица \( A' \) является матрицей линейного отображения \( A \) в базисах \( e', f' \).

Матрица \( C \) является матрицей перехода от базиса \( e \) к базису \( e' \).

Матрица \( S \) является матрицей перехода от базиса \( f \) к базису \( f' \).

Тогда матрицы \( A \) и \( A' \), представляющие одно линейное отображение в разных базисах, связаны соотношением:



\[
A' = S^{-1} A C.
\]



\subsection*{Лемма 6.1}

1) Пусть \( A_1 X = A_2 X \) для любого столбца \( X \) (где \( A_1, A_2 \) — матрицы одного размера, а \( X \) — столбец соответствующего размера). Тогда \( A_1 = A_2 \).

2) Пусть \( X A_1 = X A_2 \) для любой строки \( X \) (где \( A_1, A_2 \) — матрицы одного размера, а \( X \) — строка соответствующего размера). Тогда \( A_1 = A_2 \).

\subsection*{Следствие}

Если \( A \) — оператор, то 

\[
A' = C^{-1} A C.
\]
}