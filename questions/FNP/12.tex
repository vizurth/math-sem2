{
\subsection{Определение точек экстремума функции нескольких переменных. Необходимые и достаточные условия существования точек экстремума.}

\subsection*{Определение 6.1}

Пусть \( \Omega \subseteq \mathbb{R}^p \), \( f: \Omega \to \mathbb{R} \). Пусть \( A \) – внутренняя точка \( \Omega \).

Точка \( A \) называется **точкой максимума** (или **минимума**) функции \( f \), если существует окрестность \( U(A) \subseteq \Omega \), такая что



\[
f(X) \leq f(A) \quad (\text{или} \quad f(X) \geq f(A)), \quad \forall X \in U(A).
\]



Точки максимума и минимума функции \( f \) называются **точками экстремума** функции \( f \).

Если неравенства строгие, такие точки называются **точками строгого экстремума** функции \( f \).

\subsection*{Теорема 6.1. Необходимые условия существования экстремума}

Пусть \( A \) - точка экстремума функции \( f \), и \( \exists \frac{\partial f}{\partial x_i} (A) \). Тогда:



\[
\frac{\partial f}{\partial x_i} (A) = 0, \quad i = 1, 2, \dots, p.
\]


\subsection*{Следствие}

Пусть \( f \) дифференцируема в точке экстремума \(A\). Тогда \(d f (A) = 0\).

\subsection*{Теорема 6.2. Достаточные условия существования экстремума}

Пусть \( f \) определена и имеет непрерывные частные производные до 2-го порядка включительно в некоторой окрестности точки \( A \). Пусть \( df(A) = 0 \) (такие точки называются стационарными).

Рассмотрим второй дифференциал \( d^2f(A, \Delta x) \):



\[
d^2f(A, \Delta x) = \frac{\partial^2 f}{\partial x_1^2}(A)dx_1^2 + \frac{\partial^2 f}{\partial x_p^2}(A)dx_p^2 +
\]





\[
+ 2 \frac{\partial^2 f}{\partial x_1 \partial x_2}(A)dx_1 dx_2 + \ldots + 2 \frac{\partial^2 f}{\partial x_{p-1} \partial x_p}(A)dx_{p-1}dx_p.
\]



Рассматриваем эту величину как квадратичную форму относительно приращений \( dx_1, dx_2, \ldots, dx_p \) (здесь \( \Delta x = (dx_1, \ldots, dx_p) \)).

1) Если эта форма положительно (отрицательно) определена, то точка \( A \) является точкой строгого минимума (максимума) функции \( f \).

2) Если эта форма знакопеременная, то точка \( A \) не является точкой экстремума функции \( f \).

3) Если эта форма полуопределенная, то ничего сказать нельзя.

\subsection*{Замечание}



\[
d^2f(A; t \Delta x) = \frac{\partial^2 f}{\partial x_1^2}(A)t^2 dx_1^2 + \frac{\partial^2 f}{\partial x_p^2}(A)t^2 dx_p^2 +
\]





\[
+ 2 \frac{\partial^2 f}{\partial x_1 \partial x_2}(A)t^2 dx_1 dx_2 + \ldots + 2 \frac{\partial^2 f}{\partial x_{p-1} \partial x_p}(A)t^2 dx_{p-1}dx_p = t^2 d^2f(A; \Delta x).
\]



Значения \( d^2f(A; \Delta x) \) и \( d^2f(A; t \Delta x) \) совпадают по знаку для любого числа \( t \).


}