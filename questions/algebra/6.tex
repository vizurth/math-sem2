{
\subsection{Определение и свойства скалярного произведения. Угол между векторами.}
\subsection*{Определение 1.8}

1) Пусть \( L \) — линейное пространство над полем \( \mathbb{R} \). Пусть задана функция, сопоставляющая паре векторов \( x, y \) вещественное число, обозначаемое \( (x, y) \), и удовлетворяющая следующим требованиям:

   - \textbf{Положительная определенность}:
\[
     (x, x) \geq 0, \quad (x, x) = 0 \Leftrightarrow x = 0.
     \]

   - \textbf{Симметрия}:
\[
     (x, y) = (y, x).
     \]

   - \textbf{Линейность по первому аргументу}:
     
\[
     (\lambda x, y) = \lambda (x, y).
     \]

   - \textbf{Аддитивность}:

\[
     (x_1 + x_2, y) = (x_1, y) + (x_2, y).
\]

Тогда говорят, что в \( L \) задано \textbf{скалярное произведение}.

2) \textbf{Нормой (или длиной) вектора} \( x \) называется число:

\[
   |x| = \sqrt{(x, x)}.
\]
(Другое обозначение: \( \|x\| \).)

\subsection*{Замечания}

1) Из свойств линейности по первому аргументу и симметрии скалярного произведения следует свойство линейности по второму аргументу:

\[
( y, \alpha x + \beta z ) = \alpha ( y, x ) + \beta ( y, z ).
\]

( Требования 3) и 4) равносильны требованию 

\[
( \alpha x + \beta z , y ) = \alpha ( x, y ) + \beta ( z, y ).
\]

Следовательно,

\[
( y, \alpha x + \beta z ) = ( \alpha x + \beta z , y ) = \alpha ( y, x ) + \beta ( y, z ).
\]

2) \textbf{Нулевой элемент}: \( \langle 0, y \rangle = 0 \) для всех \( y \in L_1 \).

( Пусть \( \langle 0, y \rangle = a \). Тогда:

\[
\langle 0, y \rangle = \langle 0 +0, y \rangle = \langle 0, y \rangle + \langle 0, y \rangle.
\]

Следовательно, \( a = a + a \), откуда \( a = 0 \).)

\subsection*{Лемма 1.4: Свойства длины (нормы)}

1) \( |x| \geq 0 \), при этом \( |x| = 0 \iff x = 0 \).

2) \( |\alpha x| = |\alpha| |x| \).

3) \( |x + y| \leq |x| + |y| \) (\textbf{неравенство треугольника}).

4) \( |x y| \leq |x| |y| \) .

\subsection*{Следствие}

Пусть \( x, y \neq 0 \). Тогда выполняется неравенство:



\[
-1 \leq \frac{(x, y)}{|x| \cdot |y|} \leq 1.
\]



\subsection*{Определение 1.9}

Пусть \( x, y \neq 0 \). Так как 



\[
-1 \leq \frac{(x, y)}{|x| \cdot |y|} \leq 1,
\]



существует угол \( \varphi \in [ -\pi, \pi ] \), такой, что 



\[
\cos \varphi = \frac{(x, y)}{|x| \cdot |y|}.
\]

Этот угол называется \textbf{углом между векторами} \( x \) и \( y \).

Два вектора называются \textbf{ортогональными}, если их скалярное произведение равно нулю. Следовательно, если \( x, y \neq 0 \), то угол \( \varphi \) между ними равен \( \frac{\pi}{2} \).

Базис \( e = \{ e_1, e_2, \dots, e_n \} \) называется \textbf{ортогональным}, если:

\[
(e_i, e_j) = 0, \quad \text{при } i \neq j.
\]

Базис называется \textbf{ортонормированным}, если:

\[
(e_i, e_j) = 
\begin{cases} 
0, & i \neq j, \\ 
1, & i = j. 
\end{cases}
\]

\subsection*{Пример 1}

Пусть \( M \) — линейное пространство многочленов степени, меньшей либо равной \( n \). Введем в \( M \) скалярное произведение:



\[
(f, g) = \int_{0}^{1} f g \, dx.
\]



Данное скалярное произведение удовлетворяет всем свойствам.

Рассмотрим многочлены:



\[
f = x^2, \quad g = x^4 - \frac{3}{7}.
\]



\subsection*{Проверка ортогональности}

Вычислим скалярное произведение:



\[
(f, g) = \int_{0}^{1} x^2 \left(x^4 - \frac{3}{7} \right) \, dx.
\]



Раскрывая скобки:



\[
\int_{0}^{1} \left( x^6 - \frac{3}{7} x^2 \right) \, dx.
\]



Вычисляя интегралы:



\[
\left. \frac{x^7}{7} \right|_{0}^{1} - \left. \frac{x^3}{7} \right|_{0}^{1} = 0.
\]



Следовательно, векторы \( f \) и \( g \) **ортогональны**.

\subsection*{Вычисление нормы (длины) вектора}

Определим норму вектора \( f \):



\[
(f, f) = \int_{0}^{1} x^4 \, dx = \left. \frac{x^5}{5} \right|_{0}^{1} = \frac{1}{5}.
\]



Следовательно, длина вектора \( f \) равна:



\[
|f| = \frac{1}{\sqrt{5}}.
\]




}