{
\subsection{Частные производные функции \(p\) переменных. Связь между дифференцируемостью функции и существованием частных производных. Пример функции, которая имеет частные производные в точке \( A \), но не дифференцируема в этой точке.}

\subsection*{Определение 3.2}

Пусть \( \Omega \subset \mathbb{R}^p \); \( f: \Omega \to \mathbb{R} \). Пусть точка \( A = (a_1, a_2, ..., a_p) \) - внутренняя точка \( \Omega \).

Рассмотрим функцию одной переменной 



\[
f_i(x_i) = f(a_1, ..., a_{i-1}, x_i, a_{i+1}, ..., a_p), \quad i = 1, ..., p.
\]



(то есть, \( i \)-ю координатную функцию).  

Если эта функция имеет производную в точке \( a_i \), то эта производная называется **частной производной функции** \( f \) по переменной \( x_i \) в точке \( A \) и обозначается \( \frac{\partial f}{\partial x_i}(A) \) или \( f'_{x_i}(A) \).  

То есть,



\[
\frac{\partial f}{\partial x_i}(A) = \lim_{\Delta x_i \to 0} \frac{f(a_1, ..., a_{i-1}, a_i + \Delta x_i, a_{i+1}, ..., a_p) - f(a_1, ..., a_{i-1}, a_i, a_{i+1}, ..., a_p)}{\Delta x_i}.
\]



\subsection*{Теорема 3.2}

Если \( f \) дифференцируема в точке \( A \), то существуют все частные производные функции \( f \) в точке \( A \), и  



\[
f'(A) = (f'_{x_1}(A), f'_{x_2}(A), ..., f'_{x_p}(A)).
\]

\subsection*{Доказательство}

Имеем:



\[
f(A + \Delta X) - f(A) = f'(A, \Delta X) + o(\Delta X).
\]



Введем обозначения элементов строки \( f'(A) \): пусть \( f'(A) = (c_1, ..., c_p) \).

Рассмотрим приращение \( \Delta X = (0, ..., 0, \Delta x_i, 0, ..., 0) \) (для \( i = 1, ..., p \)).

Тогда:



\[
f(a_1, ..., a_i, ..., a_i + \Delta x_i, a_{i+1}, ..., a_p) - f(a_1, ..., a_i, a_{i-1}, a_i, a_{i+1}, ..., a_p) = c_i \Delta x_i + o(\Delta x_i).
\]



Следовательно,



\[
\frac{f(a_1, ..., a_i, ..., a_i + \Delta x_i, a_{i+1}, ..., a_p) - f(a_1, ..., a_i, a_{i-1}, a_i, a_{i+1}, ..., a_p)}{\Delta x_i} = c_i + \frac{o(\Delta x_i)}{\Delta x_i}.
\]



При \( \Delta x_i \to 0 \) имеем \( f'_{x_i}(A) = c_i \).

\subsection*{Следствие. Выражение дифференциала через частные производные}

Так как приращение \( \Delta X = (dx_1, ..., dx_p) \), то



\[
df(A) = (f'(A), \Delta X) = f'_{x_1}(A) dx_1 + f'_{x_2}(A) dx_2 + \dots + f'_{x_p}(A) dx_p.
\]



\subsection*{Замечание}

Обратное неверно: из существования всех частных производных функции \( f \) в точке \( A \) **не следует** её дифференцируемость в точке \( A \).


\subsection*{Пример}

Рассмотрим функцию \( f(x, y) = \sqrt[3]{xy} \) в точке \( A(0, 0) \).

1) Найдем частные производные функции \( f \) в точке \( A \):



\[
f'_x(0,0) = \lim_{\Delta x \to 0} \frac{f(\Delta x,0) - f(0,0)}{\Delta x} = \frac{0 - 0}{\Delta x} = 0.
\]





\[
f'_y(0,0) = 0.
\]



2) Проверим условие дифференцируемости функции в точке \( A \):



\[
f(0 + \Delta x, 0 + \Delta y) - f(0, 0) = 0 \cdot \Delta x + 0 \cdot \Delta y + o(\sqrt{\Delta x^2 + \Delta y^2}) \text{ при } (\Delta x, \Delta y) \to (0, 0).
\]



То есть:



\[
\sqrt[3]{\Delta x \Delta y} = o(\sqrt{\Delta x^2 + \Delta y^2}) \text{ при } (\Delta x, \Delta y) \to (0, 0),
\]





\[
\frac{\sqrt[3]{\Delta x \Delta y}}{\sqrt{\Delta x^2 + \Delta y^2}} \to_{(\Delta x, \Delta y) \to (0,0)} 0.
\]



Пусть \( \Delta x = \Delta y \), \( \Delta x \to 0 \). Тогда:



\[
\frac{\Delta x^{2/3}}{\sqrt{2} |\Delta x|} = \frac{1}{\sqrt{2} |\Delta x|^{1/3}} \text{ не стремится к } 0.
\]



Следовательно, \( f \) не дифференцируема в точке \( A \).


}