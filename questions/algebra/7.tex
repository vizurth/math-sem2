{
\subsection{Пространства \(R^n\) и \(R_n\)}
\subsection*{Пример. Линейные пространства \( R^n \) и \( R_{n.} \)}

\subsubsection*{1) \( n \)-мерной строкой (столбцом) называется упорядоченный набор из \( n \) вещественных чисел, записанных в строку или столбец:}



\[
X = \begin{pmatrix} x_1 \\ x_2 \\ \vdots \\ x_n \end{pmatrix}
\]



\( R^n \) — множество \( n \)-мерных столбцов:



\[
X = \begin{pmatrix} x_1 \\ x_2 \\ \vdots \\ x_n \end{pmatrix} \in R^n, \quad x_i \in \mathbb{R}, \quad i = 1, 2, \dots, n.
\]



\( R_{n.} \) — множество \( n \)-мерных строк:



\[
X = ( x_1, x_2, \dots, x_n ) \in R_{n.}, \quad x_i \in \mathbb{R}, \quad i = 1, 2, \dots, n.
\]



На множествах \( R^n \) и \( R_{n.} \) введены операции сложения и умножения на число:

- Пусть \( X = ( x_1, x_2, \dots, x_n ) \) и \( Y = ( y_1, y_2, \dots, y_n ) \).

  1) \( X = Y \quad \Leftrightarrow \quad x_i = y_i, \quad i = 1, 2, \dots, n \).
  
  2) \( Z = X + Y = ( x_1 + y_1, x_2 + y_2, \dots, x_n + y_n ) \).

     Здесь:
\[
     0 = (0, 0, \dots, 0), \quad -X = (-x_1, -x_2, \dots, -x_n).
\]

     (Аналогично для столбцов.)

  3) \( \alpha X = ( \alpha x_1, \alpha x_2, \dots, \alpha x_n ) \).

     (Аналогично для столбцов.)

Выполнены все свойства операций, следовательно, \( R^n \) и \( R_{n.} \) — линейные пространства.

\subsubsection*{2)Система векторов \( e\), состоящая из векторов  }  

\[
e_1 = (1, 0, \dots, 0), \quad e_2 = (0, 1, \dots, 0), \quad \dots, \quad e_n = (0, 0, \dots, 1),
\]

является \textbf{базисом} \( \mathbb{R}^n \), так как:

- \textbf{Порождающая система}: любой вектор \( X = (x_1, x_2, \dots, x_n) \) можно представить как линейную комбинацию:

  

\[
  X = x_1 e_1 + x_2 e_2 + \dots + x_n e_n.
  \]



- \textbf{Линейная независимость}: если \( c_1 e_1 + c_2 e_2 + \dots + c_n e_n = 0 \), то это возможно \textbf{только} при \( c_1 = c_2 = \dots = c_n = 0 \).

Этот базис называется \textbf{каноническим}, а размерность пространства \( \mathbb{R}^n \) равна \( n \), то есть \( \dim \mathbb{R}^n = n \).

Аналогичные рассуждения справедливы для \textbf{столбцов}.

\subsection*{Замечание}

Каждый столбец из \( \mathbb{R}^n \) является \textbf{своим же координатным столбцом} в каноническом базисе.

\subsubsection*{3) Скалярное произведение в \( \mathbb{R}^n \)}

Определим скалярное произведение для векторов \( X, Y \in \mathbb{R}^n \):



\[
(X, Y) = x_1 y_1 + x_2 y_2 + \dots + x_n y_n.
\]



Оно удовлетворяет всем свойствам скалярного произведения.

\subsection*{Длина (норма) вектора}

Норма (или длина) вектора \( X \) определяется как:



\[
|X| = \sqrt{x_1^2 + x_2^2 + \dots + x_n^2}.
\]



\subsection*{Ортонормированность канонического базиса}

Канонический базис является **ортонормированным**, так как:



\[
(e_i, e_j) =
\begin{cases}
0, & i \neq j, \\
1, & i = j.
\end{cases}
\]

\subsubsection*{4) Геометрическая интерпретация пространств \( \mathbb{R}_1, \mathbb{R}_2, \mathbb{R}_3 \)}

Пространство \( V_3 \) направленных отрезков (геометрических векторов) может служить геометрическим образом пространства \( \mathbb{R}_3 \). 

Для этого векторам канонического базиса \( \mathbb{R}_3 \) поставим в соответствие тройку попарно ортогональных единичных векторов:



\[
i, j, k.
\]



Тогда строке \( A = (x, y, z) \) сопоставляется геометрический вектор:



\[
a = x i + y j + z k \quad \text{из } V_3.
\]



\subsection*{Скалярное произведение}

Пусть \( B = (x_1, y_1, z_1) \), тогда строке \( B \) сопоставляется геометрический вектор:



\[
b = x_1 i + y_1 j + z_1 k \quad \text{из } V_3.
\]

Скалярное произведение:

\[
(a, b) = (A, B) = x x_1 + y y_1 + z z_1.
\]

\subsection*{Косинус угла между векторами}

\[
\cos \varphi = \frac{(a, b)}{|a| \cdot |b|} = \frac{x x_1 + y y_1 + z z_1}{\sqrt{x^2 + y^2 + z^2} \cdot \sqrt{x_1^2 + y_1^2 + z_1^2}} = \frac{(A, B)}{|A| \cdot |B|}.
\]

\subsection*{Длина (норма) вектора}

\[
|a| = \sqrt{(a, a)} = \sqrt{x^2 + y^2 + z^2} = |A|.
\]



\subsection*{Вывод}

Операции сложения, умножения на скаляр, а также скалярные произведения в \( \mathbb{R}_3 \) и \( V_3 \) соответствуют друг другу.
}