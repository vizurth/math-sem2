{
\subsection{Теорема о линейной зависимости системы из k векторов, каждый из которых является линейной комбинацией некоторой системы из m векторов (k > m).}

\subsection*{Теорема 1.1}

Пусть \( e = \{ x_1, x_2, \ldots, x_k \} \) — система векторов линейного пространства \( L \).

1. Если \( e \) содержит нулевой вектор, то \( e \) линейно зависима.

2. Пусть \( e' \subseteq e \). Тогда:

   a) если \( e' \) линейно зависима, то \( e \) линейно зависима;
   
   b) если \( e \) линейно независима, то \( e' \) линейно независима.

3. \( e \) линейно зависима \( \iff \) один из векторов является линейной комбинацией остальных.

4. Пусть \( \{ x_1, x_2, \ldots, x_{k-1} \} \) линейно независима, а \( \{ x_1, x_2, \ldots, x_{k-1}, x_k \} \) линейно зависима. Тогда \( x_k \) является линейной комбинацией векторов \( x_1, x_2, \ldots, x_{k-1} \).


\subsection*{Теорема 1.2}

Пусть \( v = \{ v_1, v_2, \ldots, v_k \} \) и \( u = \{ u_1, u_2, \ldots, u_m \} \) — две системы векторов линейного пространства \( L \). 

Если каждый вектор системы \( v \) является линейной комбинацией векторов системы \( u \) и \( k > m \), то система векторов \( v \) **линейно зависима**.


}