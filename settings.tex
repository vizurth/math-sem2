\usepackage[russian]{babel}
\usepackage[utf8]{inputenc}
\usepackage[T1,T2A]{fontenc}
\usepackage[12pt]{extsizes} % Для шрифтов
\usepackage[left=10mm, top=10mm, right=10mm, bottom=10mm, footskip=10mm]{geometry}
\usepackage{ragged2e} % Выравнивание текста
\usepackage{indentfirst} % Отступ (абзац)
\usepackage{hyperref} % Гиперссылки
\usepackage{graphicx}
\usepackage{tikz}
\usepackage{caption}
\usepackage{xcolor}
\usepackage{subcaption}
\usepackage{wrapfig}
\usepackage{pdfpages} % Пакет для работы с многостраничными PDF
\usepackage{fancyvrb}
\usepackage{float}
\usepackage{listings}
\definecolor{link_color}{HTML}{0080FF} % Цвет для гиперссылок
\usepackage{tocloft}
\usepackage{amsmath} 
\usepackage{pdfpages}
\usepackage{pgffor} % Подключение пакета
\hypersetup{
    colorlinks=true, % Гиперссылки отображаются цветными, а не в виде прямоугольников.
    linkcolor=link_color, % Цвет внутренних ссылок 
    urlcolor=link_color, % Цвет ссылок на веб-страницы.
    citecolor=link_color % Цвет ссылок на библиографию
}
\pagestyle{empty}
\lstset{
    language=C++,                % Язык программирования
    basicstyle=\ttfamily\small,  % Шрифт для кода
    numbers=left,                % Нумерация строк слева
    numberstyle=\tiny,           % Стиль шрифта для номеров строк
    stepnumber=1,                % Нумеровать каждую строку
    numbersep=5pt,               % Отступ номеров строк от кода
    showspaces=false,            % Не показывать пробелы
    showstringspaces=false,      % Не показывать пробелы в строках
    tabsize=4,                   % Размер табуляции
    breaklines=true,             % Переносить длинные строки
    breakatwhitespace=true,      % Переносить строки только по пробелам
    captionpos=b                 % Подпись снизу
}