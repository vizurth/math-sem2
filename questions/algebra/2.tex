{
\subsection{Определение линейного пространства. Теорема о линейно зависимых и независимых системах векторов.}

\subsection*{Определение 1.1 (в билете)}

Рассмотрим поле \( \mathbb{R} \) (или \( \mathbb{C} \)) и множество \( L \) некоторых математических объектов. Будем говорить, что \( L \) является \textbf{линейным (векторным) пространством} над полем \( \mathbb{R} \) (или \( \mathbb{C} \)), если введены две операции:

1. Бинарная операция \( + \) (сложение), относительно которой \( L \) образует абелеву группу.

2. Операция умножения элементов множества \( L \) на скаляры (числа) из поля \( \mathbb{R} \) (или \( \mathbb{C} \)), удовлетворяющая следующим свойствам:

   a) \( 1 \cdot x = x \quad \forall \, x \in L \);

   б) \( \alpha (\beta x) = (\alpha \beta) x \quad \forall \, x \in L, \quad \forall \, \alpha, \beta \in \mathbb{R} \) (или \( \mathbb{C} \));

   в) \( (\alpha + \beta) x = \alpha x + \beta x \quad \forall x \in L, \quad \forall \, \alpha, \beta \in \mathbb{R} \) (или \( \mathbb{C} \));

   г) \( \alpha (x + y) = \alpha x + \alpha y \quad \forall \, x, y \in L, \quad \forall \, \alpha \in \mathbb{R} \) (или \( \mathbb{C} \)).

Элементы линейного пространства \( L \) называются \textbf{векторами}.

\subsection*{Лемма 1.1}

Рассмотрим операцию умножения элементов множества \( L \) на скаляры (числа) из поля \( \mathbb{R} \) (или \( \mathbb{C} \)). Она обладает следующими свойствами:

1. \( 0 \cdot x = 0 \quad \forall x \in L \);

2. \( \alpha \cdot 0 = 0 \quad \forall \alpha \in \mathbb{R} \) (или \( \mathbb{C} \));

3. \( -x = -1 \cdot x \), где \( -x \) — противоположный вектор к \( x \);

4. \( \alpha \cdot x = 0 \iff 
\begin{cases} 
\alpha = 0 \\ 
x = 0 
\end{cases} \)
, где \( 0 \) — нейтральный элемент операции сложения в \( L \).

\subsection*{Пример}

Пусть \( M \) — множество многочленов степени, меньшей либо равной \( n \).

Это линейное пространство относительно операций сложения многочленов и умножения многочленов на число (здесь \( 0 \)-многочлен — это многочлен, равный нулю для любого \( x \), то есть, многочлен, у которого все коэффициенты равны нулю).

\subsection*{Определение 1.2}

Линейной комбинацией векторов \( x_1, x_2, \dots, x_k \) линейного пространства \( L \) называется вектор



\[
y = c_1 x_1 + c_2 x_2 + \dots + c_k x_k,
\]



где \( c_1, c_2, \dots, c_k \in \mathbb{R} \) (или \( \mathbb{C} \)).

Числа \( c_1, c_2, \dots, c_k \) называются коэффициентами линейной комбинации.

\subsection*{Лемма 1.2}

Линейная комбинация линейных комбинаций векторов \( x_1, x_2, \dots, x_k \) также является линейной комбинацией векторов \( x_1, x_2, \dots, x_k \).


\subsection*{Определение 1.3 (в билете)}

1) Система (то есть, совокупность) векторов \( x_1, x_2, \dots, x_k \) линейного пространства \( L \) называется **линейно независимой**, если равенство 



\[
c_1 x_1 + c_2 x_2 + \dots + c_k x_k = 0
\]



возможно только в случае, когда \( c_1 = c_2 = \dots = c_k = 0 \). То есть, линейная комбинация векторов \( x_1, x_2, \dots, x_k \) равна нулевому вектору **только при всех нулевых коэффициентах**.

2) Система векторов \( x_1, x_2, \dots, x_k \) называется **линейно зависимой**, если существуют числа \( c_1, c_2, \dots, c_k \), не все из которых равны нулю, такие, что 



\[
c_1 x_1 + c_2 x_2 + \dots + c_k x_k = 0.
\]



Иными словами, если хотя бы один коэффициент отличен от нуля и при этом выполняется равенство, то система является линейно зависимой.

\subsection*{Пример}

Пусть \( M \) — линейное пространство многочленов степени, меньшей либо равной \( n \). Система векторов 



\[
e = \{ 1, x, x^2, \dots, x^n \}
\]



линейно независима, поскольку равенство 



\[
c_0 + c_1 x + c_2 x^2 + \dots + c_n x^n = 0
\]



верно **только** при \( c_0 = c_1 = \dots = c_n = 0 \). Здесь \( 0 \) — многочлен, равный нулю для любого \( x \), то есть, многочлен, у которого все коэффициенты равны нулю.


}