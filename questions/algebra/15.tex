{
\subsection{Теорема Кронекера - Капелли.}

Для того, чтобы СЛАУ \( AX = B \) (\( A \) – матрица системы размера \( m \times n \), \( B \in \mathbb{R}^m \), \( X \in \mathbb{R}^n \)) была совместна (т.е. имела решения), необходимо и достаточно, чтобы \textbf{ранг матрицы \( A \)} системы был равен \textbf{рангу расширенной матрицы} системы.

(Расширенной матрицей системы называется матрица \( (A|B) \), полученная приставлением столбца \( B \) к матрице \( A \).)

При этом:
- если \( \operatorname{rank} A \) совпадает с количеством неизвестных, то \textbf{решение единственно};
- если \( \operatorname{rank} A \) меньше количества неизвестных, то \textbf{решений бесконечно много}.




\subsection*{Доказательство}

1) Пусть система совместна, т.е. существует столбец \( X \):



\[
AX = B \quad \Leftrightarrow \quad x_1 A_1 + \dots + x_n A_n = B.
\]



Следовательно, столбец \( B \) является линейной комбинацией столбцов матрицы \( A \). Добавление столбца \( B \) не увеличивает количество линейно независимых столбцов, следовательно, не меняет ранг матрицы.

2) Пусть \( \operatorname{rank} A = \operatorname{rank} (A|B) = r \). Матрица \( A \) имеет \( r \) линейно независимых столбцов, пусть это \( A_1, A_2, \dots, A_r \). Остальные столбцы, включая \( B \), являются их линейными комбинациями. Следовательно, существуют числа \( c_1, \dots, c_r \):



\[
B = c_1 A_1 + \dots + c_r A_r.
\]



Тогда вектор \( X \) имеет вид:



\[
X =
\begin{pmatrix}
c_1 \\
c_2 \\
\vdots \\
c_r \\
0 \\
\vdots \\
0
\end{pmatrix}
\]



Следовательно, столбец \( X \) является решением системы, и система совместна.

3) Пусть \( \operatorname{rank} A = \operatorname{rank} (A|B) = r \), то есть система совместна. Сведем систему к эквивалентной системе \( UX = F \) с трапециевидной матрицей \( U \) (см. метод Гаусса).

- Если \( \operatorname{rank} A = \operatorname{rank} U = n \), то количество ненулевых строк \( U \) совпадает с количеством неизвестных. Следовательно, система имеет единственное решение.

- Если \( \operatorname{rank} A = \operatorname{rank} U < n \), то количество ненулевых строк \( U \) меньше количества неизвестных, следовательно, система имеет бесконечно много решений.


}