{
\subsection{Несобственные интегралы I рода: определение, главное значение. Признаки сходимости. Сходимость интеграла}

\subsection*{Определение 3.1}

1) Пусть \( f(x) \) определена на \( [a, +\infty) \), \( f(x) \in R([a, A]) \) \( \forall A \in (a, +\infty) \).

Символ \( \int_{a}^{+\infty} f(x)dx \) называется несобственным интегралом I рода.

Если существует конечный предел 

\[
I = \lim_{A \to +\infty} \int_{a}^{A} f(x)dx,
\]

то символу \( \int_{a}^{+\infty} f(x)dx \) приписывают значение \( I \), то есть,

\[
\int_{a}^{+\infty} f(x)dx = \lim_{A \to +\infty} \int_{a}^{A} f(x)dx
\]

и говорят, что несобственный интеграл сходится.

Если предел бесконечен или не существует, то говорят, что несобственный интеграл расходится.

2) Пусть \( f(x) \) определена на \( (-\infty, b] \), \( f(x) \in R([B, b]) \) \( \forall B \in (-\infty, b) \).

Символ \( \int_{-\infty}^{b} f(x)dx \) называется несобственным интегралом I рода.

Если существует конечный предел 

\[
I = \lim_{B \to -\infty} \int_{B}^{b} f(x)dx,
\]

то символу \( \int_{-\infty}^{b} f(x)dx \) присваивают значение \( I \), то есть,

\[
\int_{-\infty}^{b} f(x)dx = \lim_{B \to -\infty} \int_{B}^{b} f(x)dx
\]

и говорят, что несобственный интеграл \textbf{сходится}.

Если предел бесконечен или не существует, то говорят, что несобственный интеграл \textbf{расходится}.

3) Пусть \( f(x) \) определена на \( (-\infty, +\infty) \), \( f(x) \) интегрируема на любом отрезке.

В этом случае символ \( \int_{-\infty}^{+\infty} f(x)dx \) также называется несобственным интегралом I рода.

Есть два равносильных способа приписать символу \( \int_{-\infty}^{+\infty} f(x)dx \) числовое значение:

\[
\text{а)} \int_{-\infty}^{+\infty} f(x)dx = \int_{-\infty}^{c} f(x)dx + \int_{c}^{+\infty} f(x)dx
\]

где \( c \) — произвольная точка из \( (-\infty, +\infty) \).

Интеграл \( \int_{-\infty}^{+\infty} f(x)dx \) сходится, если сходятся \( \int_{-\infty}^{c} f(x)dx \) и \( \int_{c}^{+\infty} f(x)dx \).

\[
\text{б)} \int_{-\infty}^{+\infty} f(x)dx = \lim_{A \to -\infty, B \to +\infty} \int_{A}^{B} f(x)dx.
\]

\subsection*{Замечание}

Если не существует конечный 



\[
\lim_{A \to +\infty} \int_{A}^{B} f(x)dx,
\]



но существует конечный 



\[
\lim_{B \to +\infty} \int_{-B}^{B} f(x)dx,
\]



то этот предел называют **главным значением интеграла** 



\[
\int_{-\infty}^{+\infty} f(x)dx
\]



и обозначают **v.p.** 



\[
\int_{-\infty}^{+\infty} f(x)dx.
\]

\[
\text{(то есть, }v.p. \int_{-\infty}^{+\infty} f(x)dx = \lim_{B \to +\infty} \int_{-B}^{B} f(x)dx).
\]

\subsection*{Примеры}


\[
\text{1) }\int_{-\infty}^{+\infty} x dx
\]
сходится только в смысле главного значения, и его главное значение равно 0.

\[
\text{2) } \int_{a}^{A} \frac{dx}{x^p}
\]

сходится при \( p > 1 \), расходится при \( p \leq 1 \).

\[
\int_{a}^{A} \frac{dx}{x^p} =
\begin{cases} 
\frac{x^{1-p}}{1-p} \Big|_{a}^{A}, & p \neq 1 \\
(\ln x) \Big|_{a}^{A}, & p = 1
\end{cases}
\]

\[
\exists \lim_{A \to \infty} \int_{a}^{A} \frac{dx}{x^p} \text{ только при }  p > 1.
\]
только при \( p > 1 \).


}