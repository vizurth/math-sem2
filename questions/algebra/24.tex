{
\subsection{Теорема об ортогональном подобии вещественной симметричной матрицы некоторой диагональной матрице. Следствия.}
\subsection*{Теорема 7.5}

Любая вещественная симметричная матрица \( A \) размера \( n \times n \) ортогонально подобна диагональной, на диагонали которой стоят собственные числа матрицы \( A \).

(То есть, существует ортогональная матрица \( Q \): \( Q^{-1}AQ = \operatorname{diag}(\lambda_1, \ldots, \lambda_n) \),

где \( \lambda_1, \ldots, \lambda_n \) — собственные числа \( A \)).

\subsection*{Доказательство}

1) Существует собственное число \( \lambda \in \mathbb{R} \) и соответствующий ему собственный вектор \( X \in \mathbb{R}^n \) матрицы \( A \) (т.е. \( A X = \lambda X \)).

Возьмем столбец \( P_1 = X / |X| \) (\( P_1 \) также собственный вектор матрицы \( A \)). Дополним \( P_1 \) векторами \( P_2', P_3', \dots, P_n' \) до базиса пространства \( \mathbb{R}^n \). Проведем процесс ортогонализации Грама–Шмидта. Получим ортонормированный базис \( P_1, P_2, \dots, P_n \) пространства \( \mathbb{R}^n \).

Рассмотрим матрицу \( P = (P_1, P_2, \dots, P_n) \). \( P \) – ортогональная матрица, так как ее столбцы – ортонормированная система векторов. Следовательно, \( P^{-1} = P^T \).

Рассмотрим матрицу \( P^{-1} A P \):



\[
P^{-1} A P = P^T A P = 
\begin{pmatrix}
P_1^T \\
P_2^T \\
\vdots \\
P_n^T
\end{pmatrix}
\begin{pmatrix}
\lambda P_1 & A P_2 & \dots & A P_n
\end{pmatrix} =
\begin{pmatrix}
\lambda P_1^T P_1 & * & \dots & * \\
\lambda P_2^T P_1 &  &  &  \\
\vdots &  & B &  \\
\lambda P_n^T P_1 &  &  &
\end{pmatrix}.
\]



Так как:

a) \( \lambda P_j^T P_1 = \lambda (P_j, P_1) = 0 \) для \( j = 2, 3, \dots, n \),

b) матрица \( P^{-1} A P \) симметричная (\( (P^T A P)^T = P^T A (P^T)^T = P^T A P \)),

то \( P^{-1} A P \) имеет вид:



\[
P^{-1} A P =
\begin{pmatrix}
\lambda & 0 & 0 & \dots & 0 \\
0 &  &  &  &  \\
0 &  & B &  &  \\
0 &  &  &  &
\end{pmatrix},
\]



где \( B \) – симметричная матрица.

2) Проведем доказательство теоремы методом математической индукции по размерности матрицы \( A \).

a) \textbf{База индукции.} Пусть \( n = 1 \), тогда \( A \) — диагональная матрица \( A = a_{11} \).

b) \textbf{Индукционный переход.} Пусть утверждение теоремы справедливо для \( n - 1 \), то есть если симметричная матрица \( B \) имеет размер \( (n - 1) \times (n - 1) \), то существует ортогональная матрица \( \tilde{Q} \) размера \( (n - 1) \times (n - 1) \), такая что:



\[
\tilde{Q}^{T} B \tilde{Q} = \operatorname{diag}(\lambda_2, \dots, \lambda_n).
\]



Рассмотрим симметричную матрицу \( A \) размера \( n \times n \). Применим преобразование, описанное в пункте 1, и получим матрицу:



\[
A' = P^{-1} A P =
\begin{pmatrix}
\lambda_1 & 0 & 0 & \dots & 0 \\
0 &  &  &  &  \\
0 &  & B &  &  \\
0 &  &  &  &
\end{pmatrix}.
\]



Так как \( B \) имеет размер \( (n - 1) \times (n - 1) \), по предположению индукции существует ортогональная матрица \( \tilde{Q} \) размера \( (n - 1) \times (n - 1) \), такая что:



\[
\tilde{Q}^{T} B \tilde{Q} = \operatorname{diag}(\lambda_2, \dots, \lambda_n).
\]


Рассмотрим матрицу 



\[
T = \begin{pmatrix}
1 & 0 & 0 & \dots & 0 \\
0 & \\
\vdots & & \tilde{Q} \\
0 &
\end{pmatrix}.
\]



Покажем, что \( T \) — ортогональная.



\[
TT^T = \begin{pmatrix}
1 & 0 & 0 & \dots & 0 \\
0 & \\
\vdots & & \tilde{Q} \\
0 &
\end{pmatrix}
\begin{pmatrix}
1 & 0 & 0 & \dots & 0 \\
0 & \\
\vdots & & \tilde{Q}^T \\
0 &
\end{pmatrix} =
\]





\[
= \begin{pmatrix}
1 & 0 & 0 & \dots & 0 \\
0 & \\
\vdots & & \tilde{Q} \tilde{Q}^T \\
0 &
\end{pmatrix} = E.
\]



Аналогично, \( T^T T = E \). (См. перемножение блочных матриц, лемма 2.2).

Рассмотрим 



\[
T^{-1}A'T = T^T A' T =
\]





\[
= \begin{pmatrix}
1 & 0 & 0 & \dots & 0 \\
0 & \\
\vdots & & \tilde{Q}^T \\
0 &
\end{pmatrix}
\begin{pmatrix}
\lambda_1 & 0 & 0 & \dots & 0 \\
0 & \\
\vdots & & B \\
0 &
\end{pmatrix}
\begin{pmatrix}
1 & 0 & 0 & \dots & 0 \\
0 & \\
\vdots & & \tilde{Q} \\
0 &
\end{pmatrix}.
\]





\[
= \begin{pmatrix}
\lambda_1 & 0 & 0 & \dots & 0 \\
0 & \\
\vdots & & \tilde{Q}^T B \\
0 &
\end{pmatrix}
\begin{pmatrix}
1 & 0 & 0 & \dots & 0 \\
0 & \\
\vdots & & \tilde{Q} \\
0 &
\end{pmatrix}
\]





\[
= \begin{pmatrix}
\lambda_1 & 0 & 0 & \dots & 0 \\
0 & \\
\vdots & & \tilde{Q}^T B \tilde{Q} \\
0 &
\end{pmatrix} = \operatorname{diag}(\lambda_1, \dots, \lambda_n).
\]

Следовательно,



\[
T^{-1} A' T = T^{-1} (P^{-1} A P) T = (P T)^{-1} A (P T) = \operatorname{diag}(\lambda_1, \dots, \lambda_n).
\]



Возьмем в качестве матрицы \( Q \) произведение \( P T \). Матрица \( Q = P T \) ортогональна, так как является произведением ортогональных матриц.

Получаем:



\[
Q^{-1} A Q = \operatorname{diag}(\lambda_1, \dots, \lambda_n).
\]



Так как собственные числа диагональной матрицы — это её элементы, стоящие на главной диагонали (выпишите характеристический многочлен диагональной матрицы и найдите его корни), и собственные числа подобных матриц совпадают (так как совпадают их характеристические многочлены), то



\[
\lambda_1, \dots, \lambda_n
\]



— собственные числа матрицы \( A \).


\subsection*{Важные следствия}

1) Столбцы матрицы \( Q \) являются собственными векторами матрицы \( A \) (см. следствие 2 к теореме 6.4).

2) Для любой симметричной матрицы \( A \) размера \( n \times n \) существуют \( n \) линейно независимых собственных векторов. Следовательно, размерность каждого собственного подпространства максимально возможная, то есть совпадает с кратностью собственного числа как корня характеристического многочлена.

3) \textbf{Построение матрицы \( Q \)}:

   * Находим собственные числа (корни характеристического многочлена).
   * Находим собственные векторы матрицы \( A \) (это линейное пространство решений однородной системы линейных алгебраических уравнений \( (A - \lambda E)X = 0 \)).
   * Выполняем ортогонализацию Грама – Шмидта базиса каждого собственного подпространства.
   * Собираем базисные векторы всех собственных подпространств и составляем из них матрицу \( Q \).

На диагонали матрицы \( Q^{-1}AQ \) будут стоять собственные числа в том порядке, в котором мы расставили соответствующие собственные векторы в матрице \( Q \).

4) Если матрица \( A \) оператора \( A \) в некотором базисе \( e \) симметричная, то существует базис \( f \), в котором матрица оператора имеет диагональный вид (возьмём \( f = e Q \), то есть, возьмём матрицу \( Q \) как матрицу перехода к базису \( f \)). Базис \( f \) состоит из собственных векторов оператора (см. лемму 6.2).

При этом, если \( e \) — ортонормированный базис, то \( f \) также будет ортонормированным базисом, так как по теореме 7.1:



\[
(f_i, f_j) = (Q_{*i}, Q_{*j}) =
\begin{cases} 
0, & i \neq j \\
1, & i = j
\end{cases}
\]



(Напоминаю, \( Q_{*i} \) — координатный столбец вектора \( f_i \) в базисе \( e \)).


}