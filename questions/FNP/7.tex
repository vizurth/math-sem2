{
\subsection{Частные производные высших порядков. Теорема о равенстве смешанных производных.}

\subsection*{Определение 4.1}

1) Пусть \( f \) определена на множестве \( \Omega \subseteq \mathbb{R}^p \), точка \( A \) - внутренняя точка \( \Omega \).

Пусть \( \exists \frac{\partial f}{\partial x_i} (X) \) в некоторой окрестности точки \( A \) (то есть, в окрестности точки \( A \) определена функция \( \frac{\partial f}{\partial x_i} (X) \)). Если \( \exists \frac{\partial}{\partial x_k} \left( \frac{\partial f}{\partial x_i} (A) \right) \), она называется частной производной второго порядка по переменным \( x_i, x_k \) функции \( f \) в точке \( A \) и обозначается \( \frac{\partial^2 f}{\partial x_k \partial x_i} (A) \) или \( f^{''}_{x_i x_k} (A) \) (возможно \( i = k \), если \( i \neq k \), производная называется смешанной).

2) Аналогично определяются частные производные \( m \)-го порядка:



\[
\frac{\partial^m f}{\partial x_i^{m_1} \partial x_{i_2}^{m_2} \ldots \partial x_{i_k}^{m_k}} (A), \quad (m_1 + m_2 + \ldots + m_k = m).
\]



\subsection*{Теорема 4.1}

Пусть \( f(x, y) \) имеет в окрестности \( U(x_0, y_0) \) \( f^{''}_{xy} \) и \( f^{''}_{yx} \), и они непрерывны в точке \( (x_0, y_0) \). Тогда они равны в точке \( (x_0, y_0) \).

\subsection*{Следствие}

Пусть \( f(x_1, ..., x_p) \) имеет в окрестности \( U(A) \subseteq \mathbb{R}_p \) все частные производные до \( k \)-го порядка включительно, и они непрерывны в точке \( A \).

Пусть \( \{i_1, i_2, ..., i_k\} \) и \( \{j_1, j_2, ..., j_k\} \) — два набора натуральных чисел из множества \( \{1, 2, ..., p\} \), отличающиеся только порядком членов. Тогда  



\[
\frac{\partial^k f}{\partial x_{i_1} \partial x_{i_2} ... \partial x_{i_k}} (A) = \frac{\partial^k f}{\partial x_{j_1} \partial x_{j_2} ... \partial x_{j_k}} (A).
\]



\subsection*{Доказательство}

От набора \"i\" к набору \"j\" можно перейти последовательными перестановками двух соседних производных. При каждом переходе используем доказанную теорему.


}