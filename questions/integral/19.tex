{
\subsection{Абсолютная и условная сходимость несобственных интегралов II рода .}
\subsection*{Определение 2.2}

Рассмотрим интеграл из пункта 2) определения 2.1 (для интегралов из пунктов 1) и 3) аналогично).




Если сходится \( \int_{a}^{b} |f(x)| dx \), то интеграл \( \int_{a}^{b} f(x) dx\) называют абсолютно \textbf{сходящимся}.

\subsection*{Теорема 2.4}

Если несобственный интеграл сходится абсолютно, то он сходится.

(без доказательства)

\subsection*{Замечание}

Обратное неверно.

Если интеграл \( \int_{a}^{b} |f(x)| dx \) расходится, а интеграл \( \int_{a}^{b} f(x) dx \) сходится, то говорят, что интеграл \( \int_{a}^{b} f(x) dx \) сходится условно.

}