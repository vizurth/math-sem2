{
\subsection{Непрерывность  функции }
\subsection*{Определение 1.4}

Пусть \( f(x) \in R([a, b]) \).

Рассмотрим функцию \( \Phi(x) = \int_{a}^{x} f(t) dt \), определенную на \( [a, b] \).

Функция \( \Phi(x) \) называется функцией верхнего предела интеграла от \( f(x) \).


\subsection*{Теорема 1.9}

Функция \( \Phi(x) \) непрерывна на \( [a, b] \).

\subsection*{Доказательство}

Зафиксируем произвольную точку \( x_0 \in [a, b] \). Тогда для любой точки \( x \in [a, b] \):

\[
\int_{a}^{x} f(t)dt = \int_{a}^{x_0} f(t)dt + \int_{x_0}^{x} f(t)dt,
\]

то есть,

\[
\Phi(x) - \Phi(x_0) = \int_{x_0}^{x} f(t)dt.
\]

Поскольку \( f(x) \in R([a, b]) \), то \( f(x) \) ограничена на \( [a, b] \), то есть,

\[
\exists K: |f(x)| \leq K \text{ на } [a, b].
\]

Тогда

\[
\left| \int_{x_0}^{x} f(t)dt \right| \leq K|x - x_0|.
\]



Следовательно,

\[
\left| \Phi(x) - \Phi(x_0) \right| \leq K|x - x_0| \xrightarrow{x \to x_0} 0.
\]

Таким образом,

\[
\lim_{x \to x_0} \Phi(x) = \Phi(x_0),
\]


то есть, \( \Phi(x) \) непрерывна в точке \( x_0 \).

Поскольку \( x_0 \) — произвольная точка отрезка \( [a, b] \), то \( \Phi(x) \) непрерывна на \( [a, b] \).

\subsection*{Теорема 1.10}

В каждой точке \( x \) промежутка \( [a, b] \), в которой \( f(x) \) непрерывна, существует \( \Phi'(x) = f(x) \).


\subsection*{Доказательство}

Зафиксируем произвольную точку \( x_0 \in [a, b] \), в которой функция непрерывна.

Возьмем произвольное \( \varepsilon > 0 \).



\[
\exists \delta > 0: (|t - x_0| < \delta \Rightarrow |f(t) - f(x_0)| < \varepsilon),
\]



то есть,



\[
f(x_0) - \varepsilon < f(t) < f(x_0) + \varepsilon, \quad \forall t \in (x_0 - \delta, x_0 + \delta).
\]



Пусть \( |\Delta x| < \delta \). Тогда на отрезке с концами в точках \( x_0 \) и \( x_0 + \Delta x \) функция \( f(t) \) удовлетворяет неравенству.

1) Тогда:

а) Пусть \( \Delta x \geq 0 \),



\[
(f(x_0) - \varepsilon) \Delta x \leq \int_{x_0}^{x_0+\Delta x} f(t) dt \leq (f(x_0) + \varepsilon) \Delta x;
\]



б) Пусть \( \Delta x < 0 \),



\[
(f(x_0) - \varepsilon)(- \Delta x) \leq \int_{x_0+\Delta x}^{x_0} f(t) dt \leq (f(x_0) + \varepsilon)(- \Delta x).
\]



Разделим все части неравенства из пункта а) на \( \Delta x \), а все части неравенства из пункта б) на \( - \Delta x \). Получим:



\[
f(x_0) - \varepsilon \leq \frac{1}{\Delta x} \int_{x_0}^{x_0+\Delta x} f(t) dt \leq f(x_0) + \varepsilon,
\]



что эквивалентно:



\[
\left| \frac{1}{\Delta x} \int_{x_0}^{x_0+\Delta x} f(t) dt - f(x_0) \right| \leq \varepsilon.
\]

2) Рассмотрим


\[
\Phi(x_0 + \Delta x) - \Phi(x_0) = \int_{a}^{x_0+\Delta x} f(t)dt - \int_{a}^{x_0} f(t)dt = \int_{x_0}^{x_0+\Delta x} f(t)dt.
\]

Следовательно,

\[
\frac{\Phi(x_0 + \Delta x) - \Phi(x_0)}{\Delta x} = \frac{1}{\Delta x} \int_{x_0}^{x_0+\Delta x} f(t)dt.
\]

3) Получили, что

\[
\forall \varepsilon > 0 \, \exists \delta > 0 : \quad ( |\Delta x| < \delta \Rightarrow \left| \frac{\Phi(x_0 + \Delta x) - \Phi(x_0)}{\Delta x} - f(x_0) \right| \leq \varepsilon).
\]

Отсюда
\[
\lim_{\Delta x\to 0} \frac{\Phi(x_0 + \Delta x) - \Phi(x_0)}{\Delta x} = f(x_0),
\]

то есть, существует \( \Phi'(x_0) = f(x_0) \).
}