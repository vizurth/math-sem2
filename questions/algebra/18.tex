{
\subsection{Линейное отображение линейных пространств. Матрица отображения в некоторых базисах. Ее использование для вычисления образа вектора.  Связь матриц отображения в разных базисах.}

\subsection*{Определение 6.1}

Отображение \( A \) линейного пространства \( V \) в линейное пространство \( W \) (\( A : V \to W \)) называется линейным, если:



\[
A(ax + by) = aAx + bAy, \quad \forall x, y \in V, \forall a, b \in \mathbb{R}.
\]


Если \( V = W \), линейное отображение \( A \) называется \textbf{линейным оператором}.

\subsection*{Примеры}

1) Отображение \( A : \mathbb{R}^n \to \mathbb{R}^m \) состоит в том, что каждый столбец \( X \in \mathbb{R}^n \) умножается слева на фиксированную матрицу \( B \) размера \( m \times n \). Отображение \( A \) линейно, так как:



\[
A (aX + bY) = B (aX + bY) = a BX + b BY = a A X + b A Y
\]





\[
\forall X, Y \in \mathbb{R}^n, \quad \forall a, b \in \mathbb{R}.
\]



2) Пусть \( M \) — линейное пространство многочленов степени \( \leq n \).



\[
A f = a_k f^{(k)} + a_{k-1} f^{(k-1)} + \dots + a_0 f
\]



— линейный оператор, сопоставляющий каждому многочлену \( f \) многочлен \( a_k f^{(k)} + a_{k-1} f^{(k-1)} + \dots + a_0 f \). Проверьте линейность самостоятельно.

\subsection*{Определение 6.2}

Пусть \( A : V \to W \) — линейное отображение.

1) Пусть \( A x = y \). Вектор \( y \in W \) называется \textbf{образом} вектора \( x \in V \).

Вектор \( x \in V \) называется \textbf{прообразом} вектора \( y \in W \).

2) Множество 

\[
A (V) = \{ y \in W \mid \exists x \in V : A x = y \}
\]



(то есть, «множество значений» отображения \( A \)) называется \textbf{образом} отображения \( A \) и обозначается \( \text{Im } A \).

3) Множество 



\[
A^{-1} (\{ 0 \}) = \{ x \in V \mid A x = 0 \}
\]

(то есть, множество прообразов вектора \( 0 \)) называется \textbf{ядром} отображения \( A \) и обозначается \( \text{Ker } A \).


}