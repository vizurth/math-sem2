{
\subsection{Евклидовы и унитарные пространства. Процесс ортогонализации Грама-Шмидта. Линейная независимость ортонормированной системы векторов}
\subsection*{Определение 7.1}

Линейное пространство над полем \( \mathbb{R} \) с заданным на нем скалярным произведением называется \textbf{евклидовым}, над полем \( \mathbb{C} \) — \textbf{унитарным}.

\subsection*{Замечание}

В унитарном пространстве \( L \) свойство симметрии скалярного произведения изменяется на:



\[
(x, y) = \overline{(y, x)}.
\]



Остальные свойства остаются прежними:

1) \( (x, x) \geq 0 \), причём \( (x, x) = 0 \iff x = 0 \).

2) \( (\alpha x + \beta y, z) = \alpha (x, z) + \beta (y, z) \quad \forall x, y, z \in L, \forall \alpha, \beta \in \mathbb{C}. \)

\subsection*{Пример 7.1}

Рассмотрим \( \mathbb{C}^n \) – линейное пространство столбцов с \( n \) комплексными компонентами. Скалярное произведение вводится следующим образом:



\[
X = \begin{pmatrix}
x_1 \\
x_2 \\
\vdots \\
x_n
\end{pmatrix}, \quad Y = \begin{pmatrix}
y_1 \\
y_2 \\
\vdots \\
y_n
\end{pmatrix}
\]



Пусть \( X, Y \) - векторы из \( \mathbb{C}^n \).



\[
(X, Y) = x_1 \overline{y_1} + x_2 \overline{y_2} + \ldots + x_n \overline{y_n} \quad (\Leftrightarrow \quad (X, Y) = X^T \overline{Y}, \quad где \quad X^T = (x_1, \ldots, x_n),
\]





\[
\overline{Y} = \begin{pmatrix}
\overline{y_1} \\
\overline{y_2} \\
\vdots \\
\overline{y_n}
\end{pmatrix} ).
\]



Тогда \( (X, X) = x_1 \overline{x_1} + x_2 \overline{x_2} + \ldots + x_n \overline{x_n} = |x_1|^2 + |x_2|^2 + \ldots + |x_n|^2 \geq 0. Выполнение остальных свойств проверьте самостоятельно.

\subsection*{Теорема 7.1}

Пусть \( E \) — евклидово пространство, \( e = \{e_1, e_2, \dots, e_n\} \) — ортонормированный базис \( E \),

(т.е. \( (e_i, e_j) = \begin{cases} 
0, & i \neq j \\
1, & i = j 
\end{cases} \)).

Пусть векторы \( x, y \) имеют координатные столбцы \( X, Y \) в базисе \( e \). Тогда:

1) \( (x, y)_e = (X, Y)_{\mathbb{R}^n} \), то есть **скалярное произведение векторов совпадает со скалярным произведением их координатных столбцов в ортонормированном базисе**.

2) Векторы \( x, y \) ортогональны \( \iff \) ортогональны их координатные столбцы в ортонормированном базисе,

т.е. \( (x, y)_e = 0 \iff (X, Y)_{\mathbb{R}^n} = 0 \).

\subsection*{Теорема 7.2. Процесс ортогонализации Грама – Шмидта}

Пусть \( f_1, f_2, \ldots, f_k \) – линейно независимая система векторов из евклидова пространства \( E \). Тогда можно построить ортонормированную систему векторов \( e_1, e_2, \ldots, e_k \), принадлежащих линейной оболочке  векторов \( f_1, f_2, \ldots, f_k \) \( (L(f_1, f_2, \ldots, f_k)) \).

\subsection*{Теорема 7.3}

1) Любая ортонормированная система векторов линейно независима.

2) В евклидовом пространстве \( E \) всегда можно построить ортонормированный базис.

\subsection*{Доказательство}

1) Пусть \( e = \{e_1, e_2, \dots, e_k\} \) — ортонормированная система векторов.

Рассмотрим равенство:



\[
c_1 e_1 + c_2 e_2 + \dots + c_k e_k = 0.
\]



Умножим обе части равенства скалярно на \( e_i \):



\[
(c_1 e_1 + c_2 e_2 + \dots + c_k e_k, e_i) = (0, e_i).
\]



Так как система ортонормирована, получим:



\[
c_i = 0, \quad i = 1, 2, \dots, k.
\]



Следовательно, равенство возможно только при \( c_1 = c_2 = \dots = c_k = 0 \), что означает линейную независимость системы.

2) Пусть \( \dim E = n \), \( f = \{f_1, f_2, \dots, f_n\} \) — базис \( E \).

Применяем процесс ортогонализации Грама–Шмидта к системе \( f \). В результате получаем ортонормированную систему векторов \( e = \{e_1, e_2, \dots, e_n\} \), принадлежащих \( E \).

Так как система \( e \) линейно независима и содержит \( n \) векторов, она образует базис пространства \( E \).


}