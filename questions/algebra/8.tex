{
\subsection{Подпространство линейного пространства. Линейная оболочка системы векторов}

\subsection*{Определение 1.10}

Подмножество \( P \) векторов линейного пространства \( L \) называется \textbf{подпространством} \( L \), если оно само является линейным пространством относительно операций, введенных в \( L \).

\subsection*{Следствие}

Подмножество \( P \) является подпространством \( L \) тогда и только тогда, когда оно \textbf{замкнуто} относительно операций сложения и умножения на число, введенных в \( L \):

\[
\forall \, x_1, x_2 \in P, \quad \forall \, a, b \in \mathbb{R} \text{ (или } \mathbb{C} \text{)} \quad ax_1 + bx_2 \in P.
\]

\subsection*{Примеры}

1) Множество столбцов из \( \mathbb{R}^n \), у которых совпадают первая и последняя компоненты, является \textbf{подпространством} \( \mathbb{R}^n \).

2) Множество столбцов из \( \mathbb{R}^n \), у которых первая компонента равна \( l \), \textbf{не является} подпространством \( \mathbb{R}^n \).

\subsection*{Определение 1.11}

Пусть \( L \) — линейное пространство, а \( u = \{ u_1, u_2, \dots, u_m \} \) — некоторая система векторов из \( L \). 

Рассмотрим множество всех возможных линейных комбинаций этих векторов:

\[
L (u_1, u_2, \dots, u_m) = \left\{ x \in L \mid x = \sum_{k=1}^{m} \alpha_k u_k \right\}.
\]


(то есть,\(L(u_1, u_2, \dots , u_m )\) - множество всех возможных линейных комбинаций векторов  \( u_1, u_2, \dots, u_m\)).

Это множество называется \textbf{линейной оболочкой} векторов \( u_1, u_2, \dots, u_m \).

\subsection*{Лемма 1.5}

1) \( L(u_1, u_2, \dots, u_m) \) — это \textbf{подпространство} \( L \).

2) Максимальный по количеству векторов \textbf{линейно независимый} набор векторов из \( u_1, u_2, \dots, u_m \) является \textbf{базисом} \( L(u_1, u_2, \dots, u_m) \).

\subsection*{Замечания}

1) \( L(u_1, u_2, \ldots, u_m) \) также называют \textbf{пространством, натянутым на векторы} \( u_1, u_2, \ldots, u_m \).

2) \textbf{Любое линейное пространство} является \textbf{линейной оболочкой} своего \textbf{базиса}.


}