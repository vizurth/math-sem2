{
\subsection{Теорема о существовании ортогонального преобразования базиса, приводящего квадратичную форму к каноническому виду. Практический метод приведения квадратичной формы к каноническому виду с помощью ортогонального преобразования базиса (метод собственных векторов).}

\subsection*{Определение 8.3}

Квадратичная форма \( B(x, x) \) в базисе \( e \) имеет \textbf{канонический вид}, если её матрица \( B \) в базисе \( e \) диагональная, то есть



\[
B = \operatorname{diag}(\lambda_1, \lambda_2, \ldots, \lambda_n).
\]



Тогда квадратичная форма принимает вид:



\[
B(x, x) = \lambda_1 x_1^2 + \lambda_2 x_2^2 + \ldots + \lambda_n x_n^2.
\]



где \( X \) — координатный столбец вектора \( x \) в базисе \( e \):



\[
X = \begin{pmatrix}
x_1 \\
x_2 \\
\vdots \\
x_n
\end{pmatrix}.
\]



Числа \( \lambda_1, \lambda_2, \ldots, \lambda_n \) называются \textbf{коэффициентами канонической формы}.

\subsection*{Теорема 8.3}

Любую вещественную квадратичную форму можно привести к каноническому виду. 

Более того, существует ортогональное преобразование базиса \( e \) в базис \( e' \), в котором квадратичная форма принимает канонический вид. В этом случае коэффициенты \( \lambda_1, \lambda_2, \dots, \lambda_n \) определяются однозначно (с точностью до порядка расположения).

\subsection*{Доказательство}

Пусть \( B \) – матрица квадратичной формы \( B(x, x) \) в базисе \( e \). \( B \) симметричная, следовательно, существует ортогональная матрица \( Q \), такая что:



\[
Q^T B Q = \operatorname{diag} (\lambda_1, \ldots, \lambda_n),
\]



где \( \lambda_1, \ldots, \lambda_n \) – собственные числа \( B \) (по теореме 7.5).

Возьмем новый базис \( e' = e Q \) (то есть матрица \( Q \) – матрица перехода к новому базису \( e' \)). Матрица \( B' \) квадратичной формы \( B(x, x) \) в базисе \( e' \) будет диагональной:



\[
B' = Q^T B Q = \operatorname{diag} (\lambda_1, \ldots, \lambda_n).
\]



Таким образом, квадратичная форма \( B(x, x) \) принимает канонический вид в базисе \( e' \).

\subsection*{Следствие}

\textbf{Практический метод приведения квадратичной формы к каноническому виду через ортогональное преобразование базиса (метод собственных векторов):}

1) Найти матрицу \( B \) квадратичной формы.
2) Определить её собственные числа \( \lambda_1, \dots, \lambda_n \).
3) Найти собственные векторы матрицы \( B \), которые составляют ортогональную матрицу \( Q \).
4) Выполнить ортогональное преобразование базиса: перейти к новому базису \( e' = e Q \), в котором квадратичная форма принимает канонический вид.


}