{
\subsection{Производная сложной функции. Частные производные сложной функции. Инвариантность формы первого дифференциала.}

\subsection*{Определение 3.3}

Пусть функции \( f_1(X), f_2(X), \dots, f_m(X) \) определены на множестве \( \Omega \subseteq \mathbb{R}^p \).

Пусть функция \( g(Y) \) определена на множестве \( D \subseteq \mathbb{R}^m \), и точка 



\[
Y = (f_1(X), f_2(X), \dots, f_m(X)) \in D, \quad \forall X \in \Omega.
\]



Тогда имеет смысл сложная функция 



\[
F(X) = g(f_1(X), f_2(X), \dots, f_m(X)),
\]



определенная на \( \Omega \).

\subsection*{Теорема 3.4}

Пусть имеет смысл сложная функция \( F(X) = g(f_1(X), f_2(X), \dots, f_m(X)) \), определенная на \( \Omega \).

Пусть функции \( f_1(X), f_2(X), \dots, f_m(X) \) дифференцируемы в точке \( X_0 \), а функция \( g(Y) \) дифференцируема в точке



\[
Y_0 = (f_1(X_0), f_2(X_0), \dots, f_m(X_0)).
\]

Тогда функция \( F(X) \) дифференцируема в точке \( X_0 \), и

\[
F'(X_0) = g'(Y_0) \begin{pmatrix} f_1'(X_0) \\ \vdots \\ f_m'(X_0) \end{pmatrix}.
\]

Где:

\[
F'(X_0) \in \mathbb{R}_p, \quad g'(Y_0) \in \mathbb{R}_m, \quad f_i'(X_0) \in \mathbb{R}_p \quad (i=1, \dots, m).
\]

\[
\begin{pmatrix} f_1'(X_0) \\ \vdots \\ f_m'(X_0) \end{pmatrix}
\]


— матрица размера \( m \times p \), обозначим её \( f'(X_0) \).

\subsection*{Следствия}

1) Частные производные сложной функции:



\[
F'(X_0) = g'(Y_0) \begin{pmatrix} f'_1(X_0) \\ \vdots \\ f'_m(X_0) \end{pmatrix}.
\]



Следовательно,



\[
\frac{\partial F}{\partial x_i}(X_0) = \left( \frac{\partial g}{\partial y_1}(Y_0), \ldots, \frac{\partial g}{\partial y_m}(Y_0) \right) \begin{pmatrix} \frac{\partial f_1}{\partial x_i}(X_0) \\ \vdots \\ \frac{\partial f_m}{\partial x_i}(X_0) \end{pmatrix} =
\]





\[
= \frac{\partial g}{\partial y_1}(Y_0) \frac{\partial f_1}{\partial x_i}(X_0) + \ldots + \frac{\partial g}{\partial y_m}(Y_0) \frac{\partial f_m}{\partial x_i}(X_0) \quad (i = 1, \ldots, p).
\]

\subsection*{Теорема 3.4}

Пусть имеет смысл сложная функция \( F(X) = g(f_1(X), f_2(X), \dots, f_m(X)) \), определенная на \( \Omega \).

Пусть функции \( f_1(X), f_2(X), \dots, f_m(X) \) дифференцируемы в точке \( X_0 \), а функция \( g(Y) \) дифференцируема в точке



\[
Y_0 = (f_1(X_0), f_2(X_0), \dots, f_m(X_0)).
\]



Тогда функция \( F(X) \) дифференцируема в точке \( X_0 \), и



\[
F'(X_0) = g'(Y_0) \begin{pmatrix} f_1'(X_0) \\ \vdots \\ f_m'(X_0) \end{pmatrix}.
\]



Где:



\[
F'(X_0) \in \mathbb{R}_p, \quad g'(Y_0) \in \mathbb{R}_m, \quad f_i'(X_0) \in \mathbb{R}_p \quad (i=1, \dots, m).
\]





\[
\begin{pmatrix} f_1'(X_0) \\ \vdots \\ f_m'(X_0) \end{pmatrix}
\]



— матрица размера \( m \times p \), обозначим её \( f'(X_0) \).

\subsection*{Замечание 3.1}

Пусть \( M, P \) - строки из \( \mathbb{R}_p \). Тогда их скалярное произведение \( (M, P) \) совпадает с произведением строки \( M \) на столбце \( P^T \), то есть, 



\[
(M, P) = M P^T.
\]


\subsection*{Частные производные сложной функции}



\[
F'(X_0) = g'(Y_0) \begin{pmatrix} f'_1(X_0) \\ \vdots \\ f'_m(X_0) \end{pmatrix}.
\]



Следовательно,



\[
\frac{\partial F}{\partial x_i}(X_0) = \left( \frac{\partial g}{\partial y_1}(Y_0), \ldots, \frac{\partial g}{\partial y_m}(Y_0) \right) \begin{pmatrix} \frac{\partial f_1}{\partial x_i}(X_0) \\ \vdots \\ \frac{\partial f_m}{\partial x_i}(X_0) \end{pmatrix} =
\]





\[
= \frac{\partial g}{\partial y_1}(Y_0) \frac{\partial f_1}{\partial x_i}(X_0) + \ldots + \frac{\partial g}{\partial y_m}(Y_0) \frac{\partial f_m}{\partial x_i}(X_0) \quad (i = 1, \ldots, p).
\]



\subsection*{Инвариантность (неизменность) формы 1-го дифференциала}

Формула



\[
dg(Y_0) = \frac{\partial g}{\partial y_1}(Y_0) dy_1(X_0) + \ldots + \frac{\partial g}{\partial y_m}(Y_0) dy_m(X_0)
\]



верна и в случае, когда переменные \( y_1, \ldots, y_m \) являются функциями:



\[
y_i = f_i(X) \quad (i = 1, \ldots, m).
\]



В этом случае



\[
Y_0 = (f_1(X_0), f_2(X_0), \ldots, f_m(X_0)).
\]




}