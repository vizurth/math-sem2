{
\subsection{Предел функции \(p\) переменных. Непрерывность функции \(p\) переменных. Теорема Вейерштрасса.}


\subsection*{Определение 2.1. Функция  }

Пусть \( \Omega \subset \mathbb{R}_p \). Отображение \( f: \Omega \to \mathbb{R} \) называется **функцией** \( p \) вещественных переменных.

\subsection*{Определение 2.2. Предельная точка (точка сгущения)}  

Пусть \( \Omega \subset \mathbb{R}_p \). Точка \( A \) называется **предельной точкой** (точкой сгущения) множества \( \Omega \), если в любой проколотой окрестности точки \( A \) содержится хотя бы одна точка из \( \Omega \).

\subsection*{Замечания} 

Точка \( A \) является предельной точкой множества \( \Omega \) тогда и только тогда, когда существует последовательность \( \{X^{(n)}\}_{n=1}^{\infty} \subset \Omega \setminus \{A\} \), такая что:



\[
X^{(n)} \to A \quad \text{при} \quad n \to \infty.
\]



(Доказательство аналогично случаю \( \mathbb{R} \).)

\subsection*{Определение 2.3. Предел функции  }

Пусть \( \Omega \subset \mathbb{R}_p \), \( f: \Omega \to \mathbb{R} \). Пусть точка \( A \) является предельной точкой множества \( \Omega \).  

Число \( b \) называется **пределом функции** \( f \) в точке \( A \) (при \( X \to A \)), если выполнено одно из двух определений:  

1) \textbf{По Коши:}  



\[
\forall \varepsilon > 0 \quad \exists \delta > 0: \quad (0 < |X - A| < \delta, \quad X \in \Omega) \Rightarrow |f(X) - b| < \varepsilon.
\]

\subsection*{Определение предела по Гейне}

2) \textbf{По Гейне: }

\[
\forall \{X^{(n)}\}_{n=1}^{\infty} \subset \Omega \setminus \{A\} : \quad (X^{(n)} \to A \text{ при } n \to \infty) \Rightarrow (f(X^{(n)}) \to b \text{ при } n \to \infty).
\]



Равносильность определений доказывается аналогично случаю \( \mathbb{R} \).  

\subsection*{Обозначения:  }



\[
\lim_{X \to A} f(X) = b \quad \text{или} \quad f(X) \to_{X \to A} b.
\]



\subsection*{Замечание:  }

Справедливы все теоремы о связи пределов с арифметическими операциями, доказанные для \( p = 1 \).


\subsection*{Определение 2.4. Непрерывность функции в точке}

Пусть \( \Omega \subset \mathbb{R}^p \), \( f: \Omega \to \mathbb{R} \). Пусть точка \( A \in \Omega \).

Функция \( f \) называется непрерывной в точке \( A \), если



\[
\forall \varepsilon > 0 \quad \exists \delta > 0: \quad (X \in U_{\delta}(A) \cap \Omega \Rightarrow |f(X) - f(A)| < \varepsilon).
\]



\subsection*{Равносильные требования:}

a) если точка \( A \) – предельная точка множества \( \Omega \), то \( \lim_{X \to A} f(X) = f(A), \)

b) если точка \( A \) – изолированная точка множества \( \Omega \), то \( f(X) \) непрерывна в точке \( A \).

\subsection*{Замечание к определению.}

Если точка \( A \) – предельная точка множества \( \Omega \), то


\[
\lim_{X \to A} f(X) = f(A) \iff \lim_{|\Delta X| \to 0} \Delta f = 0 \quad (\text{где} \quad |\Delta X| = |X - A|, \quad \Delta f = f(X) - f(A)).
\]

\subsection*{Замечания}

Справедливы теоремы об арифметических операциях над непрерывными функциями:

(Сумма, разность, произведение функций, непрерывных в точке \( A \), непрерывны в точке \( A \). Отношение функций, непрерывных в точке \( A \), непрерывно в точке \( A \), если знаменатель не обращается в 0 в точке \( A \)).

\subsection*{Примеры}

1) Функции \( x \pm y \), \( xy \) непрерывны в каждой точке \( \mathbb{R}^2 \); \( x/y \) непрерывна в каждой точке \( \mathbb{R}^2 \), кроме точек вида \( (a, 0) \).

2) Многочлен \( P(x_1, x_2, \dots, x_p) \) от \( p \) неизвестных непрерывен в каждой точке \( \mathbb{R}^p \).

3) Функция 

\[
R(x_1, x_2, \dots, x_p) = \frac{P_1(x_1, x_2, \dots, x_p)}{P_2(x_1, x_2, \dots, x_p)}
\]

непрерывна в каждой точке \( \mathbb{R}^p \), в которой \( P_2(x_1, x_2, \dots, x_p) \neq 0 \).

\subsection*{Определение 2.5}

Функция непрерывна на множестве, если она непрерывна в каждой точке множества.

\subsection*{Теорема 2.1. Вейерштрасса}

Функция, непрерывная на замкнутом ограниченном множестве, ограничена на этом множестве, а также достигает на нем своих наибольшего и наименьшего значений.

(без доказательства)

\subsection*{Замечание}

Замкнутое ограниченное множество ⇔ компакт.

\subsection*{Лемма 2.1}

Пусть \( \Omega \subset \mathbb{R}^p \); \( f: \Omega \to \mathbb{R} \). Пусть точка 



\[
A = (a_1, a_2, ..., a_p) \in \Omega.
\]



Рассмотрим функцию одной переменной 



\[
f_i(x_i) = f(a_1, ..., a_{i-1}, x_i, a_{i+1}, ..., a_p), \quad i = 1, ..., p.
\]



(Будем называть её \( i \)-ой координатной функцией.)

Если \( f \) непрерывна в точке \( A \), то \( f_i \) непрерывна в точке \( a_i \).


}