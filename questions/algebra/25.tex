{
\subsection{Определение билинейной и квадратичной форм. Матрица билинейной формы в некотором базисе, ее использование для вычисления билинейной формы. Связь матриц одной билинейной формы в разных базисах.}

\subsection*{Определение 8.1}

Пусть \( L \) — линейное пространство.

1) Функция \( B: L \times L \to \mathbb{R} \), сопоставляющая каждой паре элементов \( x, y \) из \( L \) некоторое число, называется \textbf{билинейной формой}, если \( \forall x, y, z \in L, \forall \alpha, \beta \in \mathbb{R} \) выполняются соотношения:



\[
B(\alpha x + \beta y, z) = \alpha B(x, z) + \beta B(y, z),
\]





\[
B(x, \alpha y + \beta z) = \alpha B(x, y) + \beta B(x, z).
\]



(Линейность по первому и второму аргументам.)

2) Билинейная форма называется \textbf{симметричной}, если \( \forall x, y \in L \) выполняется:



\[
B(x, y) = B(y, x).
\]



3) \textbf{Квадратичная форма} — это числовая функция \( B(x, x) \), которая получается из симметричной билинейной формы \( B(x, y) \) при \( y = x \).

\subsection*{Определение 8.2}

Пусть \( L \) — линейное пространство, \( B(x, y) \) — билинейная форма, \( e = \{e_1, e_2, \dots, e_n\} \) — базис \( L \).

**Матрицей билинейной формы** в базисе \( e \) называется матрица \( B \), элементы которой:



\[
b_{ij} = B(e_i, e_j), \quad i, j = 1, 2, \dots, n.
\]



**Матрицей квадратичной формы** \( B(x, x) \) в базисе \( e \) называется матрица соответствующей билинейной формы \( B(x, y) \).

\subsection*{Теорема 13.1}

Пусть \( L \) — линейное пространство, \( B(x, y) \) — билинейная форма, \( e = \{e_1, e_2, \dots, e_n\} \) — базис \( L \), \( B \) — матрица билинейной формы в базисе \( e \).

Пусть векторы \( x, y \) имеют координатные столбцы \( X, Y \) в базисе \( e \). Тогда:



\[
B(x, y) = X^T B Y.
\]

\subsection*{Следствия}

1) Если билинейная форма \( B(x, y) \) симметрична, то симметрична ее матрица в любом базисе 

(т.к. \( b_{ij} = B(e_i, e_j) = B(e_j, e_i) = b_{ji} \) \( (i, j = 1, 2, ..., n) \)).

Если матрица \( B \) билинейной формы \( B(x, y) \) в некотором базисе \( e \) симметрична, то \( B(x, y) \) – симметричная билинейная форма.



\[
(B(x, y) \text{ – число, следовательно, } B(x, y) = (B(x, y))^T = (X^TBY)^T = Y^TBX = B(y, x)).
\]



2) Для квадратичной формы справедливо 



\[
B(x, x) = X^TBX, \quad \text{где } B = B^T.
\]



\subsection*{Пример}

Пусть \( L \) – линейное пространство размерности 2, \( B = \begin{pmatrix} 1 & 3 \\ 3 & 4 \end{pmatrix} \) - матрица квадратичной формы в некотором базисе \( e \), 

\( X = \begin{pmatrix} x_1 \\ x_2 \end{pmatrix} \) - координатный столбец вектора \( x \) в базисе \( e \).

Тогда



\[
B(x, x) = X^TBX = (x_1, x_2) \begin{pmatrix} 1 & 3 \\ 3 & 4 \end{pmatrix} \begin{pmatrix} x_1 \\ x_2 \end{pmatrix} = 
(x_1 + 3x_2, 3x_1 + 4x_2) \begin{pmatrix} x_1 \\ x_2 \end{pmatrix} = x_1^2 + 3x_1x_2 + 3x_1x_2 + 4x_2^2 = x_1^2 + 6x_1x_2 + 4x_2^2.
\]


\subsection*{Теорема 13.2}

Пусть \( L \) – линейное пространство, \( B(x, y) \) — билинейная форма, \( e, e' \) — базисы \( L \).

Тогда матрицы \( B \) и \( B' \) билинейной формы в базисах \( e \) и \( e' \) связаны соотношением:



\[
B' = C^T B C,
\]



где \( C \) — матрица перехода от базиса \( e \) к базису \( e' \).



}