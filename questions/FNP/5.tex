{
\subsection{Дифференцируемость функции в случае существования и непрерывности частных производных. (с доказательством)}

\subsection*{Теорема 3.3}

Пусть \( \Omega \subset \mathbb{R}^p \); \( f: \Omega \to \mathbb{R} \), \( A \) – внутренняя точка \( \Omega \).

Пусть в некоторой окрестности \( U(A) \) точки \( A \) существуют все частные производные функции \( f \), и они непрерывны в точке \( A \). Тогда \( f \) дифференцируема в точке \( A \).

\subsection*{Доказательство для \( p = 2 \)}

Пусть \( A(x_0, y_0) \); приращение \( \Delta X = (\Delta x, \Delta y) \) таково, что точка



\[
(x_0 + \Delta x, y_0 + \Delta y) \in U(A).
\]



Рассмотрим \( f(x_0 + \Delta x, y_0 + \Delta y) - f(x_0, y_0) \). Используем теорему Лагранжа; \( \Theta_1, \Theta_2 \in (0, 1) \).

\[
f(x_0 + \Delta x, y_0 + \Delta y) - f(x_0, y_0) =
\]

\[
= [f(x_0 + \Delta x, y_0 + \Delta y) - f(x_0, y_0 + \Delta y)] + [f(x_0, y_0 + \Delta y) - f(x_0, y_0)] =
\]


\[
= f'_1(x_0 + \Theta_1 \Delta x, y_0 + \Delta y) \Delta x + f'_2(x_0, y_0 + \Theta_2 \Delta y) \Delta y =
\]


\[
= f'_1(x_0, y_0) \Delta x + f'_2(x_0, y_0) \Delta y + [f'_1(x_0 + \Theta_1 \Delta x, y_0 + \Delta y) - f'_1(x_0, y_0)] \Delta x +
\]

\[
+ [f'_2(x_0, y_0 + \Theta_2 \Delta y) - f'_2(x_0, y_0)] \Delta y.
\]


Осталось доказать, что 



\[
[f'_1(x_o + \Theta_1 \Delta x, y_o + \Delta y) - f'_1(x_o, y_o)] \Delta x +
\]





\[
+ [f'_2(x_o, y_o + \Theta_2 \Delta y) - f'_2(x_o, y_o)] \Delta y = o(|\Delta X|) \quad \text{при} \quad \Delta X \to 0.
\]



Действительно, 



\[
\left| \frac{\Delta x}{|\Delta X|} \right| = \frac{|\Delta x|}{\sqrt{\Delta x^2 + \Delta y^2}} \leq 1.
\]



Следовательно, 



\[
\frac{\Delta x}{|\Delta X|} \quad \text{ограничено}.
\]



Так как



\[
f'_1(x_o + \Theta_1 \Delta x, y_o + \Delta y) - f'_1(x_o, y_o) \to 0 \quad \text{при} \quad (\Delta x, \Delta y) \to (0, 0),
\]



то 



\[
[f'_1(x_o + \Theta_1 \Delta x, y_o + \Delta y) - f'_1(x_o, y_o)] \Delta x = o(|\Delta X|).
\]



Аналогично,



\[
[f'_2(x_o, y_o + \Theta_2 \Delta y) - f'_2(x_o, y_o)] \Delta y = o(|\Delta X|).
\]



\subsection*{Замечание}

Непрерывность частных производных не является необходимым условием дифференцируемости функции.





}