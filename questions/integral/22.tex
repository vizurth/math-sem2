{
\subsection{Площадь криволинейной трапеции. Вычисление площади эллипса с помощью параметризации кривой.}

\subsection*{Пример}

Найдем площадь эллипса, то есть, фигуры, ограниченной кривой



\[
\frac{x^2}{a^2} + \frac{y^2}{b^2} = 1.
\]



Введем параметризацию эллипса:



\[
\begin{cases}
x(t) = a \cos t \\
y(t) = b \sin t, \quad t \in [0, 2\pi]
\end{cases}
\]



Функции \( y_1(x) = b\sqrt{1 - \frac{x^2}{a^2}}, \quad y_2(x) = -b\sqrt{1 - \frac{x^2}{a^2}} \) - верхняя и нижняя части кривой. Тогда площадь эллипса



\[
S = \int_{-a}^{a} (y_1(x) - y_2(x))dx = \int_{-a}^{a} y_1(x)dx - \int_{-a}^{a} y_2(x)dx = [x = x(t), y = y(t) ] =  
\]





\[
= \int_{\pi}^{0} y(t)dx(t) - \int_{\pi}^{2\pi} y(t)dx(t) = -\int_{0}^{2\pi} y(t)dx(t) = ab \int_{0}^{2\pi} \sin^2 t dt
\]





\[
= \frac{ab}{2} \int_{0}^{2\pi} (1 - \cos 2t)dt = \pi ab.
\]

\subsection*{Замечание}

Мы на примере показали справедливость утверждения:

\[
\text{Если}
}
\begin{cases} 
x = x(t) \\
y = y(t) 
\end{cases}, t \in [\alpha, \beta]
\]



- уравнение гладкой замкнутой кривой без самопересечений, пробегаемой против часовой стрелки и ограничивающей слева от себя фигуру площадью \( S \), то 



\[
S = - \int_{\alpha}^{\beta} y(t) dx(t) = - \int_{\alpha}^{\beta} y(t) x'(t) dt.
\]







}