{
\subsection{20.	Собственные числа и собственные векторы оператора. Матрица оператора в базисе из собственных векторов. (с доказательсвом)
}
\subsection*{Определение 6.4}

\( A : L \to L \) — линейный оператор (\( L \) – линейное пространство).

Число \( \lambda \in \mathbb{C} \) и ненулевой вектор \( x \in L \) называются \textbf{собственным числом} и соответствующим этому числу \textbf{собственным вектором} оператора \( A \), если выполняется равенство:



\[
A x = \lambda x.
\]



\subsection*{Следствия}

1) Пусть \( e \) — базис \( L \), а \( A \) — матрица оператора \( A \) в базисе \( e \). Тогда:



\[
A x = \lambda x \iff A X = \lambda X
\]



по теореме 6.2, где \( x \) — вектор из \( L \), \( X \) — его координатный столбец в базисе \( e \), а \( \lambda X \) — координатный столбец вектора \( \lambda x \).

Следовательно, число \( \lambda \in \mathbb{C} \) и ненулевой вектор \( x \in L \) являются собственным числом и собственным вектором оператора \( A \) тогда и только тогда, когда \( \lambda \) и координатный столбец \( X \) вектора \( x \) в базисе \( e \) являются собственным числом и собственным вектором матрицы \( A \).

2) Так как матрицы \( A, A' \) оператора \( A \) в базисах \( e, e' \) связаны соотношением:



\[
A' = C^{-1} A C,
\]



где \( C \) — матрица перехода от \( e \) к \( e' \), то их \textbf{собственные числа совпадают}. Это корни характеристического многочлена:



\[
\det(A - \lambda E).
\]



Многочлен \( \phi(t) = \det(A - tE) \) называется \textbf{характеристическим многочленом} оператора \( A \).

3) \( \lambda \) — собственное число оператора \( A \) \( \iff \) \( \lambda \) — корень характеристического многочлена \( \phi(t) \).

\subsection*{Лемма 6.2}

Матрица \( A \) оператора \( A \) в базисе \( e = \{e_1, e_2, \dots, e_n\} \) имеет диагональный вид тогда и только тогда, когда базис \( e \) состоит из собственных векторов оператора. При этом на диагонали матрицы \( A \) стоят соответствующие этим векторам собственные числа оператора \( A \).

\subsection*{Доказательство}



\[
A = \operatorname{diag} (\lambda_1, \lambda_2, \dots, \lambda_n) \iff
\]





\[
\begin{pmatrix}
0 \\
\vdots \\
0 \\
\lambda_j \\
0 \\
\vdots \\
0
\end{pmatrix} = \lambda_j e_j \iff
\]



(по замечанию 6.1)



\[
A e_j = e A e_j = (e_1, e_2, \dots, e_n) \iff
\]





\[
\lambda_j \text{ - собственное число } A, \quad e_j \text{ - собственный вектор } A.
\]




}