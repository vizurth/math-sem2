{
\subsection{Объем  прямого кругового цилиндра.}
\subsection*{Определение 4.2}

Функция \( V(T) \), определенная на некотором классе \( \Omega \) множеств в пространстве, называется \textbf{объемом}, если она обладает следующими свойствами:

1) \textbf{Монотонность}: \( \forall \, T_1, T_2 \in \Omega : T_1 \subseteq T_2 \Rightarrow V(T_1) \leq V(T_2) \).

2) \textbf{Аддитивность}: Если \( T_1 \) и \( T_2 \) не имеют общих внутренних точек, то \( V(T_1 \cup T_2) = V(T_1) + V(T_2) \).

3) \textbf{Инвариантность}: Если \( T_1 \) можно совместить с \( T_2 \) при помощи параллельного переноса и поворота, то \( V(T_1) = V(T_2) \).

4) \textbf{Нормировка}: объем прямоугольного параллелепипеда  равен произведению длин его трех смежных сторон.

\subsection*{Замечание}

Множества из \( \Omega \) называются кубируемыми или измеримыми по Жордану.

Все множества, которые мы рассматриваем, измеримы (без доказательства).

\subsection*{Лемма 4.1}

Объём \( V \) прямого кругового цилиндра (тела \( T \), ограниченного поверхностью \( x^2 + y^2 = R^2 \) и плоскостями \( z = 0, z = h \)) равен

\[
V = \pi R^2 h.
\]

}