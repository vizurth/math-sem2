{
\subsection{Теорема о существовании и дифференцируемости неявно заданной функции одной переменной.}

\subsection*{Определение 5.1}

Пусть \( \Omega_1, \Omega_2 \subseteq \mathbb{R} \); \( F(x,y) \) определена на множестве \( \Omega_1 \times \Omega_2 \), и для любого \( x \in \Omega_1 \) существует единственный \( y \in \Omega_2 \) такой, что \( F(x,y)=0 \).

Тогда уравнение \( F(x,y)=0 \) определяет на множестве \( \Omega_1 \) функцию \( y=f(x) \) с множеством значений из \( \Omega_2 \) следующим образом:

каждому \( x \in \Omega_1 \) сопоставляем \( y=f(x) \), где \( F(x, f(x))=0 \).

Такая функция называется **неявно заданной** или **неявной**.

\subsection*{Пример}

Уравнение \( x^2 + y^2 - 1 = 0 \) на множестве \( [-1, 1] \times [0, 1] \) определяет функцию \( y = \sqrt{1 - x^2} \), на множестве \( [-1, 1] \times [-1, 0] \) определяет функцию \( y = -\sqrt{1 - x^2} \),

на множестве \( [-1, 1] \times [-1, 1] \) не определяет неявную функцию, так как для любого \( x \in [-1, 1] \) существуют два разных \( y \in [-1, 1] \) таких, что \( x^2 + y^2 - 1 = 0 \).

\subsection*{Теорема 5.1 существования и дифференцируемости неявно заданной функции}

1) Пусть \( F(x,y) \) определена на множестве \( [x_o - \Delta, x_o + \Delta] \times [y_o - \tilde{\Delta}, y_o + \tilde{\Delta}] \).

2) \( F(x_o, y_o) = 0 \).

3) Существуют и непрерывны \( F'_x, F'_y \) на \( [x_o - \Delta, x_o + \Delta] \times [y_o - \tilde{\Delta}, y_o + \tilde{\Delta}] \).

4) \( F'_y (x_o, y_o) \neq 0 \).

Тогда в некоторой окрестности точки \( x_o \) уравнение \( F(x,y)=0 \) определяет неявно непрерывно дифференцируемую функцию \( y = f(x) \): \( y_o = f(x_o) \).

(без доказательства).


\subsection*{Замечания}

1) Требования теоремы не являются необходимыми.

2) Если выполняются условия теоремы, то вычислить \( y'(x) \) можно следующим образом:

Функция \( y = y(x) \) такая, что \( F(x, y(x)) = 0 \) для любого \( x \) из области определения. Следовательно,



\[
\frac{dF}{dx} = 0 \iff \frac{\partial F}{\partial x} + \frac{\partial F}{\partial y} y' = 0
\]



для любого \( x \) из области определения.

Следовательно,



\[
y'(x_0) = -\frac{\frac{\partial F}{\partial x}(x_0, y_0)}{\frac{\partial F}{\partial y}(x_0, y_0)}
\]



3) Если \( F(x,y) \) имеет непрерывные частные производные 2-го порядка, то



\[
\exists y''(x) = \frac{-(F''_{x^2} + F''_{xy} y')F'_y - F'_x (F''_{xy} + F''_{y^2} y')}{(F'_y)^2}
\]



Подставив \( y'(x) = -F'_x / F'_y \), получим выражение \( y''(x) \) через частные производные 1-го и 2-го порядков функции \( F(x, y) \).

Если \( F(x,y) \) имеет непрерывные частные производные 3-го порядка, то



\[
\exists y'''(x)
\]



и выражается через частные производные функции \( F(x,y) \), и т.д.

4) Для вычисления производных функции \( y(x) \) можно использовать дифференциалы.



}