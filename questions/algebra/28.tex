{
\subsection{Теорема о существовании треугольного преобразования базиса, приводящего квадратичную форму к каноническому виду. Критерий Сильвестра.}

\subsection*{Определения}

**Определение 8.5:** Пусть дана матрица \( A \):



\[
A = \begin{pmatrix}
a_{11} & a_{12} & \cdots & a_{1n} \\
a_{21} & a_{22} & \cdots & a_{2n} \\
\vdots & \vdots & \ddots & \vdots \\
a_{n1} & a_{n2} & \cdots & a_{nn}
\end{pmatrix}
\]



Числа 



\[
\Delta_1 = a_{11}, \quad \Delta_2 = \begin{vmatrix}
a_{11} & a_{12} \\
a_{21} & a_{22}
\end{vmatrix}, \quad \ldots, \quad \Delta_k = \begin{vmatrix}
a_{11} & \cdots & a_{1k} \\
\vdots & \ddots & \vdots \\
a_{k1} & \cdots & a_{kk}
\end{vmatrix}, \quad \ldots, \quad \Delta_n = |A|
\]



называются \textbf{главными (угловыми) минорами} матрицы \( A \).

\subsection*{Определение 8.6} Преобразование базиса \( e \) (то есть переход от базиса \( e \) к базису \( f \)) называется \textbf{треугольным}, если матрица этого преобразования (то есть матрица перехода от \( e \) к \( f \)) \textbf{верхняя унитреугольная}.


\subsection*{Теорема 8.5}

Пусть \( L \) – линейное пространство, \( B(x, x) \) – квадратичная форма, определенная в \( L \), \( e = \{e_1, e_2, \ldots, e_n\} \) – базис \( L \), \( B \) - матрица квадратичной формы в базисе \( e \).

Пусть все главные миноры матрицы \( B \), кроме, возможно, последнего (\( \Delta_n = |B| \)), отличны от нуля.

Тогда существует единственное треугольное преобразование базиса \( e \), приводящее квадратичную форму к каноническому виду.

При этом коэффициенты этого канонического вида связаны с главными минорами матрицы \( B \) следующим образом:



\[
\lambda_1 = \Delta_1, \quad \lambda_2 = \frac{\Delta_2}{\Delta_1}, \quad \dots, \quad \lambda_n = \frac{\Delta_n}{\Delta_{n-1}}.
\]



(Без доказательства.)

\subsection*{Теорема 8.6. Критерий Сильвестра}

Пусть \( L \) – линейное пространство, \( B(x, x) \) – квадратичная форма, определенная в \( L \), \( e = \{e_1, e_2, \ldots, e_n\} \) – базис \( L \), \( B \) - матрица квадратичной формы в базисе \( e \).

Пусть все главные миноры матрицы \( B \), кроме, возможно, последнего (\( \Delta_n = |B| \)), отличны от нуля. Тогда:

1) Для того, чтобы \( B(x, x) \) была положительно определенной квадратичной формой, необходимо и достаточно, чтобы главные миноры матрицы \( B \) были положительными.

2) Для того, чтобы \( B(x, x) \) была отрицательно определенной квадратичной формой, необходимо и достаточно, чтобы знаки главных миноров матрицы \( B \) чередовались, и первый минор \( \Delta_1 \) был отрицательным.



}