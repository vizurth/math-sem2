{
\subsection{Интегральные суммы Римана. Определение определенного интеграла. Теорема об ограниченности функции, интегрируемой на отрезке.(c доказательством)}

\subsection*{Определение 1.1: Интегральные суммы Римана}

1) Говорят, что выбрано разбиение \( \tau = \{x_k\}_{k=0}^{n} \) отрезка \( [a, b] \), если выбраны точки \( x_0 = a, x_1, x_2, \dots, x_{n-1}, x_n = b \), такие что:


\[
x_0 < x_1 < x_2 < \dots < x_{n-1} < x_n.
\]

Длину i-го отрезка разбиения обозначим \( \Delta x_i \) (\( \Delta x_i = x_i - x_{i-1} \)). Число \( \lambda_\tau = \max \{\Delta x_1, \Delta x_2, \dots, \Delta x_n\} \) называется рангом разбиения \( \tau \).

2) Пусть функция \( f(x) \) определена на отрезке \( [a, b] \). Выберем разбиение \( \tau \) отрезка \( [a, b] \). Выберем в каждом из получившихся отрезков разбиения по точке:

\[
\xi_1 \in [x_0, x_1], \quad \xi_2 \in [x_1, x_2], \quad \dots, \quad \xi_n \in [x_{n-1}, x_n].
\]

Вычислим значение функции \( f(x) \) в этих точках и составим интегральную сумму Римана:

\[
\sigma_\tau = \sum_{k=1}^{n} f(\xi_k) \Delta x_k.
\]

3) Если существует конечный предел \( I \) интегральных сумм при стремлении ранга разбиения к нулю, и этот предел не зависит ни от выбора разбиения \( \tau \), ни от выбора точек \( \xi_1, \xi_2, \dots, \xi_n \), то этот предел называют определенным интегралом от функции \( f(x) \) по отрезку \( [a, b] \) и обозначают:

\[
\int_a^b f(x)dx.
\]

То есть, 


\[
\lim_{\lambda_{\tau} \to 0} \sigma_{\tau}
\]


то есть,

\[
\forall \, \varepsilon > 0 \, \exists \, \delta > 0: \, \left( \lambda_{\tau} < \delta \implies |\sigma_{\tau} - I| < \varepsilon \right) \, \forall \, \tau, \, \forall \, \{ \xi_{k} \}_{k=0}^{n}
\]

\subsection*{Замечания}

1) \( \lambda_\tau \to 0 \Rightarrow n \to \infty \). Обратное неверно.

2) Геометрический смысл \( \sigma_\tau \) для \( f(x) \geq 0 \).

\subsection*{Определение 1.2}

Если существует \( \int_a^b f(x)dx \), то говорят, что \( f(x) \) интегрируема на отрезке \( [a, b] \) и пишут \( f(x) \in R([a, b]) \) (читается: \( f(x) \) принадлежит классу функций, интегрируемых на отрезке \( [a, b] \)).

\subsection*{Теорема 1.1}

Если \( f(x) \) интегрируема на отрезке \( [a, b] \), то \( f(x) \) ограничена на \( [a, b] \).

\subsection*{Замечание}

Обратное неверно.

\subsection*{Пример}

Функция Дирихле:

\[
f(x) =
\begin{cases} 
1, & x \in \mathbb{Q} \\ 
0, & x \notin \mathbb{Q}
\end{cases}
\]

где \( \mathbb{Q} \) — множество рациональных чисел.
}