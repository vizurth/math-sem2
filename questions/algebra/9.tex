{
\subsection{Ортогональные матрицы}

\subsection*{Определение 2.1}

Квадратная матрица \( Q \in \mathbb{R}^{n \times n} \) называется **ортогональной**, если:



\[
Q Q^{T} = Q^{T} Q = E.
\]



\subsection*{Замечание}

Матрица \( Q \) ортогональна \( \iff \) существует обратная матрица \( Q^{-1} \), равная транспонированной:



\[
Q^{-1} = Q^{T}.
\]



\subsection*{Лемма 2.1}

1) Матрица \( Q \) является ортогональной \( \iff \) её \textbf{столбцы} образуют \textbf{ортонормированную систему} векторов в \( \mathbb{R}^{n} \).

2) Матрица \( Q \) является ортогональной \( \iff \) её \textbf{строки} образуют \textbf{ортонормированную систему} векторов в \( \mathbb{R}^{n} \).

\subsection*{Замечание }

Пусть \( X, Y \in \mathbb{R}^n \), тогда их скалярное произведение определяется как:



\[
(X, Y) = x_1 y_1 + x_2 y_2 + \dots + x_n y_n = X^T Y.
\]



Здесь \( X \) и \( Y \) записаны в виде столбцов:



\[
X = \begin{pmatrix} x_1 \\ x_2 \\ \vdots \\ x_n \end{pmatrix}, \quad Y = \begin{pmatrix} y_1 \\ y_2 \\ \vdots \\ y_n \end{pmatrix}.
\]




Это определение скалярного произведения удовлетворяет всем его свойствам.

\subsection*{Лемма 2.2}

1) \( Q \) — ортогональная матрица \( \iff \) \( Q^T \) — ортогональная матрица.

2) Пусть \( Q_1 \) и \( Q_2 \) — ортогональные матрицы одного размера. Тогда \( Q_1 Q_2 \) — ортогональная матрица.

\subsection*{Доказательство}

1) а) Пусть \( Q \) — ортогональная матрица. Тогда:

\[
Q^T (Q^T)^T = Q^T Q = E, \quad Q (Q^T)^T = Q Q^T = E.
\]

Следовательно, \( Q^T \) — ортогональная матрица.

б) Пусть \( Q^T \) — ортогональная матрица. Тогда:

\[
Q Q^T = (Q^T)^T Q^T = E, \quad Q^T Q = Q^T (Q^T)^T = E.
\]

Следовательно, \( Q \) — ортогональная матрица.

2) Пусть \( Q_1 \) и \( Q_2 \) — ортогональные матрицы одного размера. Тогда:

\[
(Q_1 Q_2) (Q_1 Q_2)^T = (Q_1 Q_2) (Q_2^T Q_1^T) = Q_1 Q_2 Q_2^T Q_1^T = Q_1 Q_1^T = E.
\]

\[
(Q_1 Q_2)^T (Q_1 Q_2) = (Q_2^T Q_1^T) (Q_1 Q_2) = Q_2^T Q_1^T Q_1 Q_2 = Q_2^T Q_2 = E.
\]



Следовательно, \( Q_1 Q_2 \) — ортогональная матрица.

\subsection*{Лемма 2.3}

Пусть \( X, Y \in \mathbb{R}^n \), \( Q \,\,\,\,  {n \times n} \) — ортогональная матрица. Тогда:

1) \( \langle Q X, Q Y \rangle = \langle X, Y \rangle \).

2) \( |Q X| = |X| \)
}