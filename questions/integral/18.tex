{
\subsection{Первый и второй признаки сравнения. Сходимость интеграла}
\subsection*{Теорема 2.2}

Первый признак сравнения несобственных интегралов II рода от неотрицательных функций.

Рассмотрим интеграл из пункта 2) определения 2.1 (для интегралов из пунктов 1) и 3) аналогично).

Пусть \( f(x) \geq g(x) \geq 0 \) на \( [a, b) \). Тогда:

1) Если \( \int_{a}^{b} f(x)dx \) сходится, то \( \int_{a}^{b} g(x)dx \) тоже сходится.

2) Если \( \int_{a}^{b} g(x)dx \) расходится, то \( \int_{a}^{b} f(x)dx \) тоже расходится.

\subsection*{Теорема 2.3}

Второй признак сравнения несобственных интегралов II рода от неотрицательных функций.

Рассмотрим интеграл из пункта 2) определения 2.1 (для интегралов из пунктов 1) и 3) аналогично).

Пусть \( f(x), g(x) > 0 \) на \([a, b)\), и 

\[
\lim_{x \to b-0} \frac{f(x)}{g(x)} = l, \quad l \neq 0, l \neq \infty
\]

(например, \( f(x) \sim g(x) \)). Тогда

\[
\int_{a}^{b} f(x) dx \quad \text{сходится} \quad \iff \quad \int_{a}^{b} g(x) dx \quad \text{сходится}.
\]

\subsection*{Примеры}

Пусть \( p > 0 \).



\[
\int_{a}^{b} \frac{dx}{(b-x)^p}, \quad \int_{a}^{b} \frac{dx}{(x-a)^p}, \quad \int_{a}^{b} \frac{dx}{x^p}
\]



Сходятся при \( p < 1 \), расходятся при \( p \geq 1 \).



\[
\int_{a}^{b} \frac{dx}{x^p} = 
\begin{cases} 
\frac{x^{1-p}}{1-p} \Big|_{a}^{b}, & p \neq 1 \\
\ln(x) \Big|_{a}^{b}, & p = 1
\end{cases}
\]





\[
\exists \text{ существует} \lim_{a \to +0} \int_{a}^{b} \frac{dx}{x^p} \text{ только при } p < 1
\]

\subsection*{Примеры интегралов}


\[
\text{a)} \int_{0}^{1} \frac{dx}{\sqrt{1-x^4}} \text{ сходится.} 
\]

\[
\text{б) }\int_{0}^{2} \frac{dx}{x^3 \sqrt{x + x^4}} \text{ расходится.} 
\]

}