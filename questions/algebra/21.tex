{
\subsection{Линейная независимость собственных векторов, соответствующих различным собственным числам оператора. Собственные подпространства, их размерность. Следствия.}

\subsection*{Определение 6.4}

\( A : L \to L \) — линейный оператор (\( L \) – линейное пространство).

Число \( \lambda \in \mathbb{C} \) и ненулевой вектор \( x \in L \) называются \textbf{собственным числом} и соответствующим этому числу \textbf{собственным вектором} оператора \( A \), если выполняется равенство:



\[
A x = \lambda x.
\]



\subsection*{Следствия}

1) Пусть \( e \) — базис \( L \), а \( A \) — матрица оператора \( A \) в базисе \( e \). Тогда:



\[
A x = \lambda x \iff A X = \lambda X
\]



по теореме 6.2, где \( x \) — вектор из \( L \), \( X \) — его координатный столбец в базисе \( e \), а \( \lambda X \) — координатный столбец вектора \( \lambda x \).

Следовательно, число \( \lambda \in \mathbb{C} \) и ненулевой вектор \( x \in L \) являются собственным числом и собственным вектором оператора \( A \) тогда и только тогда, когда \( \lambda \) и координатный столбец \( X \) вектора \( x \) в базисе \( e \) являются собственным числом и собственным вектором матрицы \( A \).

2) Так как матрицы \( A, A' \) оператора \( A \) в базисах \( e, e' \) связаны соотношением:



\[
A' = C^{-1} A C,
\]



где \( C \) — матрица перехода от \( e \) к \( e' \), то их \textbf{собственные числа совпадают}. Это корни характеристического многочлена:



\[
\det(A - \lambda E).
\]



Многочлен \( \phi(t) = \det(A - tE) \) называется \textbf{характеристическим многочленом} оператора \( A \).

3) \( \lambda \) — собственное число оператора \( A \) \( \iff \) \( \lambda \) — корень характеристического многочлена \( \phi(t) \).

\subsection*{Лемма 6.3}

Собственные векторы матрицы \( A \), соответствующие различным собственным числам, линейно независимы.

\subsection*{Теорема 6.4}

1) Собственные векторы оператора, отвечающие различным собственным числам, линейно независимы.

2) Собственные векторы оператора, отвечающие одному собственному числу \( \lambda \), объединенные с нулевым вектором, образуют линейное подпространство пространства \( L \). Это подпространство называется \textbf{собственным подпространством}, отвечающим (соответствующим) собственному числу \( \lambda \).

3) Размерность собственного подпространства, отвечающего собственному числу \( \lambda \), не превосходит кратности собственного числа \( \lambda \) как корня характеристического многочлена.

\subsection*{Важные следствия}

1) Матрица \( A \) оператора \( A \) в некотором базисе \( e = \{e_1, e_2, \dots, e_n\} \) имеет диагональный вид тогда и только тогда, когда \( e \) состоит из собственных векторов оператора.

Это возможно, когда все собственные числа оператора вещественны, и размерность каждого собственного подпространства максимально возможная, то есть совпадает с кратностью собственного числа как корня характеристического многочлена (так как мы должны набрать \( n \) линейно независимых собственных векторов, которые получим, объединив базисы собственных подпространств).

2) Пусть \( A \) — квадратная матрица размера \( n \times n \).

Пусть существует невырожденная матрица \( C \) размера \( n \times n \):



\[
C^{-1}AC = \operatorname{diag} (\lambda_1, \lambda_2, \dots, \lambda_n).
\]



Тогда столбцы \( C \) — это собственные векторы матрицы \( A \), соответствующие собственным числам \( \lambda_1, \lambda_2, \dots, \lambda_n \).

Действительно, рассмотрим матрицу \( A \) как матрицу некоторого линейного оператора в некотором базисе \( e \) (см. замечание 6.2). Тогда матрицу \( C \) можно рассматривать как матрицу перехода к новому базису. В новом базисе матрица оператора диагональная, следовательно, новый базис состоит из собственных векторов оператора. Следовательно, столбцы матрицы \( C \), которые являются координатными столбцами векторов нового базиса в исходном базисе \( e \), это собственные векторы матрицы \( A \) (см. следствие 1 и определение 6.4).

3) Квадратная матрица \( A \) диагонализируема (т. е. подобна диагональной, т. е. существует невырожденная матрица \( C \), такая что:



\[
C^{-1}AC = \operatorname{diag} (\lambda_1, \lambda_2, \dots, \lambda_n)
\]



тогда и только тогда, когда все собственные числа матрицы вещественны, и размерность каждого собственного подпространства максимально возможная, то есть совпадает с кратностью собственного числа как корня характеристического многочлена.

(Собственные векторы матрицы, отвечающие одному собственному числу \( \lambda \), объединённые с нулевым вектором, образуют линейное подпространство пространства \( \mathbb{R}^n \), доказательство аналогично теореме 6.4, пункт 2).

 

}