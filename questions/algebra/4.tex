{
\subsection{Базис линейного пространства. Теорема об инвариантности числа элементов базиса. Теорема о количестве элементов линейно независимой системы (Т. 1.3, Т.1.4).}

\subsection*{Определение 1.4}

Система векторов \( u = \{ u_1, u_2, \ldots, u_m \} \) называется **порождающей** для линейного пространства \( L \), если любой вектор из \( L \) можно представить в виде линейной комбинации векторов системы \( u \).

\subsection*{Определение 1.5}

Упорядоченная система векторов \( u = \{ u_1, u_2, \ldots, u_m \} \) называется **базисом** линейного пространства \( L \), если она:
1) линейно независима;
2) порождающая для линейного пространства \( L \).

\subsection*{Замечание}

Базис линейного пространства определяется неоднозначно.

\subsection*{Пример}

Пусть \( M \) — линейное пространство многочленов степени, меньшей либо равной \( n \).

Система векторов \( e = \{ 1, x, x^2, \ldots, x^n \} \) является базисом \( M \).

\subsection*{Теорема 1.3}

Количество элементов базиса является **инвариантом**, то есть неизменным, для линейного пространства \( L \).

\subsection*{Определение 1.6}

Пусть \( u = \{ u_1, u_2, \ldots, u_m \} \) является базисом линейного пространства \( L \). Количество \( m \) элементов базиса называется **размерностью** линейного пространства \( L \) (обозначение: \( \dim L = m \)).

Говорят, что \( L \) — линейное пространство размерности \( m \) или \( m \)-мерное линейное пространство.

Если не существует базиса, состоящего из конечного числа элементов, пространство называется **бесконечномерным**.

\subsection*{Теорема 1.4}

Пусть \( L \) — линейное пространство (в дальнейшем будем писать ЛП) и \( \dim L = n \).

Пусть система векторов \( u = \{ u_1, u_2, \ldots, u_m \} \) линейно независима. Тогда выполняются следующие свойства:

1) \( m \leq n \);

2) если \( m = n \), то \( u \) является базисом \( L \);

3) если \( m < n \), то \( u \) можно дополнить до базиса векторами из \( L \).


}