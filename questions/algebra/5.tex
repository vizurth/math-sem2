{
\subsection{Координаты вектора. Теоремы о координатах вектора (Т.1.5 и Т.   .1). ???????}

\subsection*{Определение 1.7}
Пусть \( L \) – линейное пространство, а \( e = \{e_1, e_2, \dots, e_n\} \) – его базис.

Каждый вектор \( x \) из \( L \) можно представить в виде линейной комбинации векторов базиса:

\[
x = c_1 e_1 + c_2 e_2 + \dots + c_n e_n,
\]

где \( c_1, c_2, \dots, c_n \) – координаты вектора \( x \) в базисе \( e \).


\[
\text{Запись координатного столбца: }
X = \begin{pmatrix}
c_1 \\
c_2 \\
\vdots \\
c_n
\end{pmatrix}
\]
Этот столбец называется \textbf{координатным столбцом вектора} \( x \) в базисе \( e \).

\subsection*{Теорема 1.5}

Координаты вектора в базисе определяются \textbf{однозначно}.

\subsection*{Доказательство}

Пусть \( x = c_1 e_1 + c_2 e_2 + \dots + c_n e_n \) и \( x = a_1 e_1 + a_2 e_2 + \dots + a_n e_n \).

Тогда:

\[
0 = (c_1 - a_1) e_1 + (c_2 - a_2) e_2 + \dots + (c_n - a_n) e_n.
\]

Так как система векторов \( e \) \textbf{линейно независима}, то:

\[
c_1 - a_1 = c_2 - a_2 = \dots = c_n - a_n = 0,
\]

то есть:

\[
c_1 = a_1, \quad c_2 = a_2, \quad \dots, \quad c_n = a_n.
\]
}