{
\subsection{Связь между линейной зависимостью системы векторов и соответствующей системы координатных столбцов. Связь координатных столбцов одного вектора в разных базисах.}

\subsection*{Теорема 5.1. Действия с векторами в координатной форме}

Пусть \( L \) — линейное пространство, \( e = \{ e_1, e_2, \dots, e_n \} \) — базис \( L \).

Пусть векторам \( x, y, z \) сопоставлены координатные столбцы:



\[
X = \begin{pmatrix} x_1 \\ x_2 \\ \vdots \\ x_n \end{pmatrix}, \quad Y = \begin{pmatrix} y_1 \\ y_2 \\ \vdots \\ y_n \end{pmatrix}, \quad Z = \begin{pmatrix} z_1 \\ z_2 \\ \vdots \\ z_n \end{pmatrix}
\]



в базисе \( e \).

Тогда равенство \( z = a x + b y \), где \( a, b \in \mathbb{R} \), равносильно равенству:



\[
Z = a X + b Y,
\]



то есть:



\[
\begin{pmatrix} z_1 \\ z_2 \\ \vdots \\ z_n \end{pmatrix} = \begin{pmatrix} a x_1 + b y_1 \\ a x_2 + b y_2 \\ \vdots \\ a x_n + b y_n \end{pmatrix}.
\]


\subsection*{Теорема 5.2}

Векторы \( x_1, x_2, \dots, x_k \) и их координатные столбцы \( X_1, X_2, \dots, X_k \) в некотором базисе линейно зависимы или независимы одновременно.

\subsection*{Определение 5.1}

Пусть \( L \) — линейное пространство, \( e = \{e_1, e_2, \dots, e_n\} \), \( e' = \{e'_1, e'_2, \dots, e'_n\} \) — базисы \( L \).

**Матрицей перехода** от базиса \( e \) к базису \( e' \) называется матрица \( C \), столбцами которой являются координатные столбцы векторов базиса \( e' \) в базисе \( e \).

\subsection*{Замечание 10.1}



\[
X = \begin{pmatrix} x_1 \\ x_2 \\ \vdots \\ x_n \end{pmatrix}
\]



1) \( X \) — координатный столбец вектора \( x \) в базисе \( e \) \( \Leftrightarrow \)



\[
x = x_1 e_1 + \dots + x_n e_n \Leftrightarrow x = (e_1, e_2, \dots, e_n) \begin{pmatrix} x_1 \\ x_2 \\ \vdots \\ x_n \end{pmatrix} \Leftrightarrow x = eX.
\]



(матричное умножение базисной строки \( e = (e_1, e_2, \dots, e_n) \) на координатный столбец \( X \)).



2)Аналогично, \[
e'_j = (e_1, e_2, \dots, e_n) \cdot (c_{1j}, c_{2j}, \dots, c_{nj})(\text{столбец}}) = e \cdot C_{*j}, \quad \text{где } C_{*j} \text{ — } j\text{-й столбец матрицы } C. 
\]


3)Следовательно,



\[
e' = (e'_1, e'_2, \dots, e'_n) = (e \cdot C_{*1}, e \cdot C_{*2}, \dots, e \cdot C_{*n}) = e \cdot C.
\]



(матричное умножение базисной строки \( e = (e_1, e_2, \dots, e_n) \) на матрицу \( C \)).

\subsection*{Теорема 5.3}

\textbf{Связь координат одного вектора в разных базисах}

Пусть \( L \) — линейное пространство, \( e = \{e_1, e_2, \dots, e_n\} \), \( e' = \{e'_1, e'_2, \dots, e'_n\} \) — базисы \( L \).

Пусть вектору \( x \) сопоставлены координатные столбцы:



\[
X =
\begin{pmatrix}
x_1 \\
x_2 \\
\vdots \\
x_n
\end{pmatrix},
\quad
X' =
\begin{pmatrix}
x'_1 \\
x'_2 \\
\vdots \\
x'_n
\end{pmatrix}
\]



в базисах \( e \) и \( e' \).

Пусть \( C \) — матрица перехода от базиса \( e \) к базису \( e' \).

Тогда:



\[
X = C X'.
\]


\subsection*{Замечание}

Матрица \( C \) невырожденная, так как её столбцы линейно независимы по теореме 5.2. Следовательно,



\[
X' = C^{-1} X.
\]



}