{
\section{Алгебра}
\subsection{Группы, кольца, поля.}
\subsection*{Определение:}

Пусть \( A \) - множество математических объектов одной природы, на котором задано отображение 



\[
f: (A \times A) \to A,
\]



то есть, правило, сопоставляющее каждой паре элементов \( (a, b) \), \( a \in A \), \( b \in A \), некоторый элемент \( c \in A \). 

Тогда говорят, что на множестве \( A \) введена бинарная операция. 

Обозначение: 


\[
a \circ b = c.
\]

\subsection*{Определение:}

Если на множестве \( A \) введена бинарная операция, обладающая свойствами 2, 3, 4, то множество \( A \) называется \textbf{группой}.

Если также выполнено свойство 1, группа называется \textbf{абелевой}.

\subsection*{Примеры}

1) \( \mathbb{Z} \) - абелева группа относительно операции сложения.

2) \( \mathbb{Q}^+ = \{q \in \mathbb{Q} \mid q > 0\} \) - абелева группа относительно операции умножения 
(здесь \(«0» = 1, «-q» = \frac{1}{q} \)).

\subsection*{Возможные свойства бинарных операций}

1. \textbf{Коммутативность}:


\[
a \circ b = b \circ a
\]

2. \textbf{Ассоциативность}:

\[
(a \circ b) \circ c = a \circ (b \circ c)
\]



3. \textbf{Существование нейтрального элемента} операции, называемого нулем (обозначаемого \( 0 \)), то есть, элемента, не меняющего второй элемент, участвующий в операции. То есть,

\[
a \circ 0 = 0 \circ a = a \quad \forall a \in A
\]

4. \textbf{Существование противоположного элемента} для каждого элемента \( a \) множества \( A \), обозначаемого \( -a \), такого, что

\[
a \circ (-a) = (-a) \circ a = 0
\]

\subsection*{Определение 4}

Пусть \( A \) — абелева группа относительно операции \( \circ \). Пусть на \( A \) задана ещё одна бинарная операция \( * \).

Если для всех \( a, b, c \in A \) выполняются \textbf{распределительные свойства}:

5.
\[
(a \circ b) * c = (a * c) \circ (b * c)
\]

\[
a * (b \circ c) = (a * b) \circ (a * c)
\]

то множество \( A \) называется \textbf{кольцом}, а \( \circ \) — сложением, \( * \) — умножением.

\subsection*{Определение 5}

Пусть \( A \) — абелева группа относительно операции \( \circ \). Пусть на \( A \) задана ещё одна бинарная операция \( * \), обладающая следующими свойствами:

6. \textbf{Коммутативность}: 


\[
a * b = b * a
\]



7. \textbf{Ассоциативность}: 


\[
(a * b) * c = a * (b * c)
\]



8. \textbf{Существование нейтрального элемента} операции, называемого единицей (обозначаемого \( I \)), то есть, элемента, не меняющего второй элемент, участвующий в операции:


\[
a * I = I * a = a, \quad \forall a \in A
\]



9. \textbf{Существование обратного элемента} для каждого \( a \neq 0 \) множества \( A \), обозначаемого \( 1/a \), такого, что:


\[
a * (1/a) = (1/a) * a = I
\]



Тогда множество \( A \) называется \textbf{скалярным полем} или \textbf{полем}.

\subsection*{Примеры}

\( \mathbb{R} \), \( \mathbb{C} \) — поля действительных и комплексных чисел.

}