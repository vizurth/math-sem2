{
\subsection{Интегрирование рациональных дробей.}

\subsection*{Теорема 2.1: Интегрирование правильных рациональных дробей вида \( \frac{A}{(x-a)^k} \)}
\setcounter{equation}{0}
\begin{equation}
\int \frac{A}{x-a} \,dx = \int \frac{A}{x-a} \,d(x-a) = A \ln |x-a| + c.
\end{equation}

\begin{equation}
\int \frac{A}{(x-a)^k} \,dx = \int \frac{A}{(x-a)^k} \,d(x-a) = \frac{A}{(1-k)(x-a)^{k-1}} + c \quad (k \neq 1).
\end{equation}

\subsection*{Теорема 2.2: Интегрирование правильных рациональных дробей вида \( \frac{Bx + C}{(x^2 + px + q)^n} \)}

Выделим полный квадрат из квадратного трехчлена:


\[
x^2 + px + q = (x + (p/2))^2 + q^*, \quad \text{где} \quad q^* = q - p^2/4 > 0, \quad \text{так как} \quad p^2 - 4q < 0.
\]

Сделаем замену \( t = x + (p/2) \), тогда \( x = t - (p/2) \) и \( dx = dt \).

Следовательно,


\[
\int \frac{Bx + C}{(x^2 + px + q)^n} dx = \int \frac{B(t - (p/2)) + C}{(t^2 + q^*)^n} dt =
\]

\[
= \int \frac{Bt}{(t^2 + q^*)^n} dt + \int \frac{C^*}{(t^2 + q^*)^n} dt, \quad \text{где} \quad C^* = - B(p/2) + C.
\]


Разберем, как вычисляются интегралы \( \int \frac{t}{(t^2 + q^*)^n} dt \) и \( \int \frac{1}{(t^2 + q^*)^n} dt \). После вычисления интегралов следует заменить \( t \) на \( x + \frac{p}{2} \).



\[
\text{a)}\int \frac{t}{(t^2 + q^*)^n} dt = \frac{1}{2} \int \frac{2t \, dt}{(t^2 + q^*)^n} = \frac{1}{2} \int \frac{d(t^2 + q^*)}{(t^2 + q^*)^n}
\]

\[
\Large = \begin{cases} 
\frac{1}{2} \ln(t^2 + q^*) + c, & n = 1 \\ 
\frac{1}{2(1 - n)(t^2 + q^*)^{1 - n}} + c, & n \neq 1 
\end{cases}
\]

\subsection*{Вычисление интеграла \( I_n \)}

Рассмотрим интеграл:

\[
I_n = \int \frac{1}{(t^2 + q^*)^n} dt.
\]

Для случая \( n = 1 \):

\[
I_1 = \int \frac{1}{t^2 + q^*} dt = \frac{1}{\sqrt{q^*}} \arctg \frac{t}{\sqrt{q^*}} + C.
\]

Для \( n \geq 2 \), используем метод интегрирования по частям:

\[
I_n = \int \frac{1}{(t^2 + q^*)^n} dt = \left[ u = (t^2 + q^*)^{-n}, \quad dv = dt \right].
\]

Тогда:

\[
du = -n(t^2 + q^*)^{-n-1} 2t dt, \quad v = t.
\]

\[
I_n = \frac{t}{(t^2 + q^*)^n} + 2n \left[ \int \frac{t^2}{(t^2 + q^*)^{n+1}} dt \right].
\]

Учитывая, что \( t^2 = (t^2 + q^*) - q^* \), преобразуем:

\[
I_n = \frac{t}{(t^2 + q^*)^n} + 2n I_n - 2nq^* I_{n+1}.
\]

Получаем рекуррентную формулу:

\[
I_{n+1} = \frac{1}{2nq^*} \left[ \frac{t}{(t^2 + q^*)^n} + (2n-1) I_n \right].
\]

Эта формула позволяет вычислять \( I_2, I_3, \ldots \) последовательно.

\subsection*{Определение 1.2: Определенный интеграл как предел интегральных сумм}

Пусть \( f(x) \) ограничена на отрезке \( [a, b] \). Рассмотрим разбиение \( \tau = \{x_k\}_{k=0}^{n} \) этого отрезка:

\[
a = x_0 < x_1 < x_2 < \dots < x_n = b.
\]

Определим интегральную сумму:

\[
\sigma_\tau = \sum_{k=1}^{n} f(\xi_k) \Delta x_k,
\]

где \( \xi_k \) — произвольные точки в \( [x_{k-1}, x_k] \), а \( \Delta x_k = x_k - x_{k-1} \).

Если существует конечный предел интегральных сумм при стремлении \( \lambda_\tau \) (ранга разбиения) к нулю, то этот предел называют определенным интегралом функции \( f(x) \) на \( [a, b] \):

\[
I = \lim_{\lambda_\tau \to 0} \sigma_\tau = \int_a^b f(x)dx.
\]

}