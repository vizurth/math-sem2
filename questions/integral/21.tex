{
\subsection{Признак Дирихле. Сходимость интеграла при р>0.}

\subsection*{Определение 3.2}

Рассмотрим интеграл из определения 3.1.



\[
\int_{a}^{+\infty} |f(x)| dx
\]



Если сходится \( \int_{a}^{+\infty} f(x) dx \), то интеграл называют **абсолютно сходящимся**.

\subsection*{Теорема 3.1}

Признак Дирихле сходимости несобственного интеграла I рода.

Рассмотрим интеграл \( \int_{a}^{+\infty} f(x)g(x)dx \). Пусть:

1) \( f(x) \in C([a, +\infty)) \) и имеет ограниченную первообразную на \( [a, +\infty) \);

2) \( g(x) \in C^1([a, +\infty)) \), \( g(x) \) монотонно убывает на \( [a, +\infty) \) и \( \lim_{x \to +\infty} g(x) = 0 \).

Тогда интеграл \( \int_{a}^{+\infty} f(x)g(x)dx \) \textbf{сходится}.

\subsection*{Пример}
Интеграл 
\[
\int_{a}^{+\infty} \frac{\sin x}{x^p} dx
\]
(при \( a > 0, p > 0 \)) **сходится абсолютно** при \( p > 1 \), **сходится условно** при \( 0 < p \leq 1 \).



}