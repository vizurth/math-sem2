{
\subsection{Горизонтальный и вертикальный ранги матрицы. Ранг по минорам. Их совпадение для трапециевидной матрицы(с доказательством)}

\subsection*{Теорема 4.1}

Столбцы (строки) квадратной матрицы \( A \) \textbf{линейно независимы} \( \iff \) \( |A| \neq 0 \).

\subsection*{Доказательство}

1) Пусть \( |A| \neq 0 \). Докажем от противного: 

Если строки матрицы \( A \) являются линейно зависимой системой в \( \mathbb{R}_n \), то одна из строк является линейной комбинацией остальных. Пусть 



\[
A_{k*} = \alpha_1 A_{1*} + \dots + \alpha_{k-1} A_{(k-1)*} + \alpha_{k+1} A_{(k+1)*} + \dots + \alpha_n A_{n*}.
\]



Тогда:



\[
|A| = (k)
\begin{vmatrix}
A_{1*} \\
\vdots \\
A_{k*} \\
\vdots \\
A_{n*}
\end{vmatrix}
=
\alpha_1
\begin{vmatrix}
A_{1*} \\
\vdots \\
A_{k*} \\
\vdots \\
A_{n*}
\end{vmatrix} 
+ \dots + 
\alpha_n
\begin{vmatrix}
A_{1*} \\
\vdots \\
A_{k*} \\
\vdots \\
A_{n*}
\end{vmatrix} = 0
\]

так как в каждом слагаемом определитель имеет две одинаковые строки.  Противоречие. 
Для столбцов доказательство аналогично.


2) Пусть столбца матрицы \(A\) линейно независимы. То есть, \(\sum_{j=1}^n{c_j A_{*j} = 0}\) только при

\[
c_1 = \dots = c_n = 0.\,\,\, \text{То есть, система }
\begin{cases}
с_{1}a_{11} + c_2a_{12} + \dots + c_n a_{1n} = 0, \\
\dots \dots \dots \dots\dots\dots\dots\dots\dots\dots \\
с_{1}a_{n1} + c_2a_{n2} + \dots + c_n a_{nn} = 0,
\end{cases}
\]
имеет единственное решение. Следовательно, по теореме Крамера,    \(|A| \neq 0\)  .
Доказательство для строк аналогично.

\subsection*{Определение 4.1}

Пусть \( A \) — матрица размера \( m \times n \).

1) Пусть \( L_\text{г}(A) \) — линейная оболочка строк матрицы \( A \). \textbf{Горизонтальным рангом} матрицы \( A \) называется размерность этого линейного пространства:



\[
r_\text{г}(A) = \dim L_\text{г}(A).
\]



2) Пусть \( L_\text{в}(A) \) — линейная оболочка столбцов матрицы \( A \). \textbf{Вертикальным рангом} матрицы \( A \) называется размерность этого линейного пространства:



\[
r_\text{в}(A) = \dim L_\text{в}(A).
\]



\subsection*{Следствие}

Совокупность строк матрицы является \textbf{порождающей системой} пространства \( L_2(A) \). Максимальный по количеству векторов \textbf{линейно независимый} набор строк является \textbf{базисом} \( L_2(A) \) (см. Лемму 1.5). Следовательно, \textbf{горизонтальный ранг матрицы} равен количеству линейно независимых строк.

Аналогично, \textbf{вертикальный ранг матрицы} равен количеству линейно независимых столбцов.

\subsection*{Определение 4.2}

1) \textbf{Минором} матрицы \( A \) называется \textbf{определитель} квадратной матрицы, полученной из \( A \) путем вычеркивания некоторого количества строк и столбцов. Размер минора — это количество его строк (столбцов).

2) \textbf{Ранг матрицы по минорам} \( r_m(A) \) — это \textbf{наибольший размер отличного от нуля минора} этой матрицы.


\subsection*{Теорема 4.2}

Пусть \( U \) — трапециевидная матрица размера \( m \times n \). Тогда ее \textbf{вертикальный}, \textbf{горизонтальный} ранги и \textbf{ранг по минорам} совпадают и равны количеству \textbf{ненулевых строк} \( U \).

\subsection*{Доказательство}

Рассмотрим матрицу \( U \):



\[
U =
\begin{bmatrix}
u_{11} & * & * & \dots & * & * \\
0 & u_{22} & * & \dots & * & * \\
0 & 0 & u_{33} & \dots & * & * \\
\vdots & \vdots & \vdots & \ddots & * & * \\
0 & 0 & 0 & \dots & u_{r,r+1} & u_{r,n}  \\
0 & 0 & 0 & \dots & 0 & 0  \\
\vdots & \vdots & \vdots & \vdots & \vdots & \vdots \\
0 & 0 & 0 & \dots & 0 & 0
\end{bmatrix}
\]



где элементы \( u_{11}, u_{22}, \dots, u_{rr} \) не равны нулю, элементы \( u_{r,r+1}, \dots, u_{rn} \), а также элементы, стоящие на месте \(*\), могут быть любыми.

Построим квадратную \textbf{невырожденную} матрицу \( U^{(1)} \), выделяя ненулевые строки:



\[
U^{(1)} =
\begin{bmatrix}
u_{11} & * & * & * & * \\
0 & u_{22} & * & * & * \\
0 & 0 & u_{33} & * & * \\
0 & 0 & 0 & \ddots & * \\
0 & 0 & 0 & 0 & u_{rr}
\end{bmatrix}
\]



Так как \( |U^{(1)}| \neq 0 \), это означает, что количество линейно независимых строк \( U \) равно количеству её \textbf{ненулевых строк}. 

Следовательно, \textbf{вертикальный ранг}, \textbf{горизонтальный ранг} и \textbf{ранг по минорам} матрицы \( U \) совпадают и равны количеству \textbf{ненулевых строк} \( U \).







\subsection*{Доказательство}

1) Любой минор матрицы \( U \) размера, большего, чем \( r \), равен \( 0 \), так как содержит нулевую строку. Следовательно, \( r_M(U) = r \).

2) Столбцы матрицы \( U^{(1)} \) линейно независимы (см. теорему 4.1). Их количество равно \( r \), поэтому они образуют базис пространства \( \mathbb{R}^r \). Следовательно, каждый столбец матрицы \( U^{(2)} \) является линейной комбинацией столбцов матрицы \( U^{(1)} \).

\[
U^{(2)} =
\begin{bmatrix}
* & \dots & *  \\
* & \dots & *  \\
* & \dots & *  \\
* & \dots & * & \\
u_{r,r+1} & \dots & u_{rm} & 
\end{bmatrix}
\]

\subsection*{Дополнение матрицы}

Дополняем столбцы матриц \( U^{(1)} \) и \( U^{(2)} \) нулями до столбцов матрицы \( U \).

Первые \( r \) столбцов матрицы \( U \) остаются линейно независимыми, так как:

\[
c_1 U_{r+1} + c_2 U_{r+2} + \dots + c_r U_{r*} = 0 \Rightarrow c_1 U^{(1)}_{r+1} + c_2 U^{(1)}_{r+2} + \dots + c_r U^{(1)}_{r*} = 0 \Rightarrow c_1 = \dots = c_r = 0.
\]

Столбцы матрицы \( U \) с номерами \( r+1, \dots, n \) продолжают быть линейными комбинациями первых \( r \) столбцов (так как добавленные элементы равны нулю).

Следовательно, \( r_6(U) = r \).

\subsection*{Доказательство}

3) Рассмотрим матрицу \( \overline{U} \):



\[
\overline{U} =
\begin{pmatrix}
U^{(1)} \\
U^{(2)}
\end{pmatrix}.
\]



Соединим \( U^{(1)} \) и \( U^{(2)} \) (не перемножим, а приставим друг к другу, получим новую матрицу, состоящую из первых \( r \) строк матрицы \( U \)).

Строки матрицы \( \overline{U} \) \textbf{линейно независимы}, так как:



\[
c_1 \overline{U}_{1*} + c_2 \overline{U}_{2*} + \dots + c_r \overline{U}_{r*} = 0
\]





\[
\Rightarrow c_1 U^{(1)}_{1*} + c_2 U^{(1)}_{2*} + \dots + c_r U^{(1)}_{r*} = 0
\]





\[
\Rightarrow c_1 = c_2 = \dots = c_r = 0.
\]



Поскольку строки \( U^{(1)} \) \textbf{линейно независимы} (так как \( |U^{(1)}| \neq 0 \)), остальные строки матрицы \( U \) являются \textbf{нулевыми}.

Следовательно, матрица \( \overline{U} \) имеет \( r \) линейно независимых строк, то есть:



\[
r_\text{г}(U) = r.
\]




}