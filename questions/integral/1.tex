{
\section{Интегралы}
\subsection{Определение и свойства первообразной. Теорема о связи первообразных одной функции.}
\subsection*{Определение 1.1.}

Пусть функция \( f(x) \) определена на промежутке \( \langle a, b \rangle \). Функция \( F(x) \), определенная на промежутке \( \langle a, b \rangle \), называется \textbf{первообразной} функции \( f(x) \) на \( \langle a, b \rangle \), если 

\[
F'(x) = f(x), \quad \forall \, x \in \langle a, b \rangle.
\]


На концах промежутка имеем в виду односторонние производные функции \( F(x) \).


\subsection*{Следствие.}

Если \( F(x) \) является первообразной некоторой функции на \( \langle a, b \rangle \), то \( F(x) \) непрерывна на \( \langle a, b \rangle \).

\subsection*{Теорема о
связи первообразных одной функции}

Пусть \( F(x) \) — первообразная функции \( f(x) \) на \( \langle a, b \rangle \). Тогда:

1. \( \forall \, c \in \mathbb{R} \quad F(x) + c \) также первообразная функции \( f(x) \) на \( \langle a, b \rangle \).
2. Если \( \Phi(x) \) — некоторая первообразная функции \( f(x) \) на \( \langle a, b \rangle \), то \( \exists \, c \in \mathbb{R} \colon \Phi(x) = F(x) + c \).

\subsection*{Доказательство}

1. \( (F(x) + c)' = f(x) \).
2. \( (\Phi(x) - F(x))' = f(x) - f(x) = 0 \) на \( \langle a, b \rangle \). Следовательно, \( \Phi(x) - F(x) = c \) на \( \langle a, b \rangle \).

\subsection*{Следствие}

Если \( F(x) \) — некоторая первообразная функции \( f(x) \) на \( \langle a, b \rangle \), то каждая функция семейства функций \( \{ F(x) + c \} \quad (c \in \mathbb{R}) \) является \textbf{первообразной}, и других первообразных нет.

}