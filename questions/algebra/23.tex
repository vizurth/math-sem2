{
\subsection{Теорема о собственных числах и собственных векторах вещественной симметричной матрицы.}
\subsection*{Теорема 7.4}

1) Все собственные числа вещественной симметричной матрицы вещественны.

2) Собственные векторы вещественной симметричной матрицы, соответствующие разным собственным числам, ортогональны.

\subsection*{Доказательство}

1) Пусть \( \lambda \) — собственное число матрицы \( A \), столбец \( X \in \mathbb{C}^n \) — собственный вектор матрицы, соответствующий собственному числу \( \lambda \).

a) Рассмотрим число \( \alpha = \overline{X}^T A X \).



\[
\overline{\alpha} = \alpha^T = (\overline{X}^T A X)^T = \dot{X}^T \dot{A} \dot{X} = \overline{X}^T A X = \alpha.
\]



Следовательно, \( \alpha \) — вещественное число.

(В первом переходе используем: \( (DBC)^T = \dot{C}^T \dot{B}^T \dot{D}^T \), а также то, что \( A^T = A \). Во втором переходе используем то, что \( \dot{X}^T \dot{A} \dot{X} \) — число, слагаемые которого являются произведениями элементов столбцов \( \dot{X}, X \) и матрицы \( A \). Пользуемся свойствами комплексного сопряжения: \( a + b = \overline{a} + \overline{b} \), \( ab = \overline{a} \cdot \overline{b} \). В результате каждый элемент столбцов \( \dot{X}, X \) меняется на комплексно сопряженный, элементы матрицы \( A \) не меняются, так как они вещественны.)

b) \( \alpha = \overline{X}^T A X = \overline{X}^T \lambda X = \lambda |X|^2 \), где число \( |X|^2 = \overline{X}^T X = (\overline{X}^T {X})^T = {X}^T \overline{X} \) — квадрат длины столбца \( X \). Следовательно,



\[
\lambda = \frac{\alpha}{|X|^2}
\]



— вещественное число.

2) Пусть \( \lambda_1, \lambda_2 \) — собственные числа матрицы \( A \) (\( \lambda_1 \neq \lambda_2 \)), столбцы \( X_1, X_2 \in \mathbb{R}^n \) — собственные векторы матрицы, соответствующие собственным числам \( \lambda_1, \lambda_2 \).

а) Покажем, что \( (A X_1, X_2) = (X_1, A X_2) \).



\[
(A X_1, X_2) = (A X_1)^T X_2 = X_1^T A X_2 = X_1^T (A X_2) = (X_1, A X_2).
\]



б) Следовательно,



\[
0 = (A X_1, X_2) - (X_1, A X_2) = (\lambda_1 X_1, X_2) - (X_1, \lambda_2 X_2) = (\lambda_1 - \lambda_2) (X_1, X_2).
\]



Так как \( \lambda_1 \neq \lambda_2 \), то \( (X_1, X_2) = 0 \), что доказывает ортогональность собственных векторов, соответствующих различным собственным числам.

\subsection*{Замечание}

Так как \( \lambda \) — вещественное собственное число вещественной симметричной матрицы \( A \), рассматриваем только вещественные собственные векторы \( X \) матрицы, соответствующие собственному числу \( \lambda \), которые являются решениями СЛАУ:



\[
(A - \lambda E) X = 0.
\]




}