{
\subsection{Теорема об интегрировании функции, равной нулю всюду, за исключением конечного числа точек, и функции, у которой изменены значения в конечном числе точек.}

\subsection*{Теорема 1.5}

1) Пусть \( f(x) \) определена и ограничена на \( [a, b] \) и равна нулю всюду, за исключением конечного числа точек. Тогда


\[
\int_{a}^{b} f(x) dx = 0.
\]



2) Пусть \( g(x) \in R([a, b]) \).

Если в конечном числе точек изменить значения функции \( g(x) \), то функция останется интегрируемой, и величина интеграла не изменится.

}