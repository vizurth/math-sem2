{
\subsection{Теорема о необходимом и достаточном условии положительной (отрицательной) определенности квадратичной формы.}

\subsection*{Определение 8.4}

Пусть \( L \) – линейное пространство, \( B(x, x) \) – квадратичная форма, определенная в \( L \).

1) Квадратичная форма \( B(x, x) \) называется \textbf{положительно (отрицательно) определенной}, если \( B(x, x) > 0 \) (\( B(x, x) < 0 \)) для любого ненулевого вектора \( x \) из линейного пространства \( L \).

Такие квадратичные формы называются \textbf{знакоопределенными}.

2) Квадратичная форма \( B(x, x) \) называется \textbf{знакопеременной}, если \( \exists x, y \in L \) такие, что \( B(x, x) > 0 \) и \( B(y, y) < 0 \).

3) Квадратичная форма \( B(x, x) \) называется \textbf{положительно (отрицательно) полуопределенной}, если \( B(x, x) \geq 0 \) (\( B(x, x) \leq 0 \)) \( \forall x \in L \),

и существует ненулевой вектор \( x^* \in L \): \( B(x^*, x^*) = 0 \).

Такие квадратичные формы называются \textbf{полуопределенными (квазиопределенными)}.

\subsection*{Следствие}

\textbf{Скалярное произведение} – это симметричная билинейная форма, причем соответствующая ей квадратичная форма \textbf{положительно определена}.

\subsection*{Теорема 8.4}

Пусть \( L \) — линейное пространство, \( B(x, x) \) — квадратичная форма, определенная в \( L \).

1) Квадратичная форма \( B(x, x) \) является \textbf{положительно (отрицательно) определенной} тогда и только тогда, когда все коэффициенты её канонического вида положительны (отрицательны).

2) Квадратичная форма \( B(x, x) \) является \textbf{положительно (отрицательно) полуопределенной} тогда и только тогда, когда все коэффициенты её канонического вида неотрицательны (неположительны), и существует хотя бы один коэффициент, равный нулю.

3) Квадратичная форма \( B(x, x) \) является \textbf{знакопеременной} тогда и только тогда, когда среди коэффициентов её канонического вида есть хотя бы один положительный и хотя бы один отрицательный.

\subsection*{Замечание}

Справедливо утверждение (закон инерции квадратичных форм):

Каким бы способом мы ни привели квадратичную форму к каноническому виду, количество положительных, отрицательных и нулевых коэффициентов останется неизменным.

}